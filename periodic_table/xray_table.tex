%%
%%  Periodic Table of X-ray absorption edges and emission lines
%%
%%  using data from Elam, Ravel, and Sieber,
%%  Radiation Physics and Chemistry 63 (2002)
%%
%%  latex sources based on periodic_table  from Ivan Griffin,
%%
%%  modified by Matt Newville, Jan-2013
%%
%%
%% This work may be distributed and/or modified under the
%% conditions of the LaTeX Project Public License, either version 1.3
%% of this license or (at your option) any later version.
%% The latest version of this license is in
%%   http://www.latex-project.org/lppl.txt
%% and version 1.3 or later is part of all distributions of LaTeX
%% version 2005/12/01 or later.
%%
%%%%%%%%%%%%%%%%%%%%%%%%%%%%%%%%%%%%%%%%%%%%%%%%%
\documentclass[]{article}

\usepackage{verbatim}
\usepackage{tikz}
\usepackage[active,tightpage]{preview}

\usetikzlibrary{shapes,calc,arrows}


\begin{document}
\begin{preview}

%% Note on Size:
%%   with scale=1.00, the poster made is (roughly)
%%     127.5 cm x  61 cm    (~50.4 in x 24.2 in)
%%
%%  To make a poster of a different size, set scale appropriately:
%%      scale=0.22  for letter size paper (11in width)
%%      scale=0.78  for 1 meter width
%%      scale=0.714 for 36 inch width

\begin{tikzpicture}[font=\sffamily, scale=0.714, transform shape]

%% Fill Color Styles
  \tikzstyle{ElementFill} = [fill=yellow!4]
  \tikzstyle{Element} = [draw=black, ElementFill,
  minimum width=70mm, minimum height=70mm, node distance=70mm]

  \tikzstyle{SpaceFill} = [fill=white]
  \tikzstyle{Space} = [draw=white, SpaceFill,
  minimum width=5mm, minimum height=5mm, node distance=5mm]

  \tikzstyle{TitleLabel} = [font={\sffamily\Huge\bfseries}, scale=3.50]
  \tikzstyle{SubTitleLabel} = [font={\Huge}, scale=2.50]

  \definecolor{MedRed}{rgb}{0.8,0,0}
  \definecolor{MedBlue}{rgb}{0,0,0.8}
  \definecolor{DeepBlue}{rgb}{0,0,0.6}

  \newcommand{\Color}[2]{{\textcolor{#1}{#2}}}
  \newcommand{\BRed}[1]{{\Color{MedRed}{\textbf{#1}}}}
  \newcommand{\BBlue}[1]{{\Color{MedBlue}{\textbf{#1}}}}
  \newcommand{\Red}[1]{{\Color{MedRed}{#1}}}
  \newcommand{\Blue}[1]{{\Color{MedBlue}{#1}}}
  \newcommand{\Name}[1]{{\Color{DeepBlue}{\textsf{\textbf{#1}}}}}

  %%  Hydrogen
\newcommand{\ElemH}{{%
  \begin{minipage}{67mm}%
  \vspace{1mm}

  {\Huge{\hspace{1mm} {\textbf{H}} \hfill \hfil{\textbf{1}} \hspace{1mm}}} %

  \vspace{6mm}

  {\Huge{\hfill {\Name{hydrogen}} \hfill}}

  \vspace{6mm}

  {\Large{
  \begin{tabular*}{67mm}%
   {@{\hspace{5pt}}{r}@{\extracolsep{\fill}}r@{\extracolsep{\fill}}r}%
   %\multicolumn{3}{@{\hspace{1pt}}c}{ }\\%
   %\multicolumn{3}{c}{{\Huge{\BBlue{hydrogen}}} }\\%
   % \multicolumn{3}{@{\hspace{1pt}}c}{ }\\%
   {\BRed{14}}  & {\bf{ }} &  {\bf{ }} \\%
   {\BBlue{ }} &   &   \\%
   {\BBlue{ }} &   &   \\%
   {\BRed{ }} & {\bf{ }} & {\bf{ }} \\%
   {\BBlue{ }} &   &   \\%
   \multicolumn{3}{@{\hspace{1pt}}c}{ }\\%
   \end{tabular*}
   \vfill}}
  \end{minipage}}}

%%  Helium
\newcommand{\ElemHe}{{%
  \begin{minipage}{67mm}%
  \vspace{1mm}

  {\Huge{\hspace{1mm} {\textbf{He}} \hfill \hfil{\textbf{2}} \hspace{1mm}}} %

  \vspace{6mm}

  {\Huge{\hfill {\Name{helium}} \hfill}}

  \vspace{6mm}

  {\Large{
  \begin{tabular*}{67mm}%
   {@{\hspace{5pt}}{r}@{\extracolsep{\fill}}r@{\extracolsep{\fill}}r}%
   %\multicolumn{3}{@{\hspace{1pt}}c}{ }\\%
   %\multicolumn{3}{c}{{\Huge{\BBlue{helium}}} }\\%
   % \multicolumn{3}{@{\hspace{1pt}}c}{ }\\%
   {\BRed{25}}  & {\bf{ }} &  {\bf{ }} \\%
   {\BBlue{ }} &   &   \\%
   {\BBlue{ }} &   &   \\%
   {\BRed{ }} & {\bf{ }} & {\bf{ }} \\%
   {\BBlue{ }} &   &   \\%
   \multicolumn{3}{@{\hspace{1pt}}c}{ }\\%
   \end{tabular*}
   \vfill}}
  \end{minipage}}}

%%  Lithium
\newcommand{\ElemLi}{{%
  \begin{minipage}{67mm}%
  \vspace{1mm}

  {\Huge{\hspace{1mm} {\textbf{Li}} \hfill \hfil{\textbf{3}} \hspace{1mm}}} %

  \vspace{6mm}

  {\Huge{\hfill {\Name{lithium}} \hfill}}

  \vspace{6mm}

  {\Large{
  \begin{tabular*}{67mm}%
   {@{\hspace{5pt}}{r}@{\extracolsep{\fill}}r@{\extracolsep{\fill}}r}%
   %\multicolumn{3}{@{\hspace{1pt}}c}{ }\\%
   %\multicolumn{3}{c}{{\Huge{\BBlue{lithium}}} }\\%
   % \multicolumn{3}{@{\hspace{1pt}}c}{ }\\%
   {\BRed{55}}  & {\bf{ }} &  {\bf{ }} \\%
   {\BBlue{5}} &   &   \\%
   {\BBlue{ }} &   &   \\%
   {\BRed{ }} & {\bf{ }} & {\bf{ }} \\%
   {\BBlue{ }} &   &   \\%
   \multicolumn{3}{@{\hspace{1pt}}c}{ }\\%
   \end{tabular*}
   \vfill}}
  \end{minipage}}}

%%  Beryllium
\newcommand{\ElemBe}{{%
  \begin{minipage}{67mm}%
  \vspace{1mm}

  {\Huge{\hspace{1mm} {\textbf{Be}} \hfill \hfil{\textbf{4}} \hspace{1mm}}} %

  \vspace{6mm}

  {\Huge{\hfill {\Name{beryllium}} \hfill}}

  \vspace{6mm}

  {\Large{
  \begin{tabular*}{67mm}%
   {@{\hspace{5pt}}{r}@{\extracolsep{\fill}}r@{\extracolsep{\fill}}r}%
   %\multicolumn{3}{@{\hspace{1pt}}c}{ }\\%
   %\multicolumn{3}{c}{{\Huge{\BBlue{beryllium}}} }\\%
   % \multicolumn{3}{@{\hspace{1pt}}c}{ }\\%
   {\BRed{112}}  & {\bf{109}} &  {\bf{ }} \\%
   {\BBlue{8}} &   &   \\%
   {\BBlue{3}} &   &   \\%
   {\BRed{3}} & {\bf{ }} & {\bf{ }} \\%
   {\BBlue{ }} &   &   \\%
   \multicolumn{3}{@{\hspace{1pt}}c}{ }\\%
   \end{tabular*}
   \vfill}}
  \end{minipage}}}

%%  Boron
\newcommand{\ElemB}{{%
  \begin{minipage}{67mm}%
  \vspace{1mm}

  {\Huge{\hspace{1mm} {\textbf{B}} \hfill \hfil{\textbf{5}} \hspace{1mm}}} %

  \vspace{6mm}

  {\Huge{\hfill {\Name{boron}} \hfill}}

  \vspace{6mm}

  {\Large{
  \begin{tabular*}{67mm}%
   {@{\hspace{5pt}}{r}@{\extracolsep{\fill}}r@{\extracolsep{\fill}}r}%
   %\multicolumn{3}{@{\hspace{1pt}}c}{ }\\%
   %\multicolumn{3}{c}{{\Huge{\BBlue{boron}}} }\\%
   % \multicolumn{3}{@{\hspace{1pt}}c}{ }\\%
   {\BRed{188}}  & {\bf{183}} &  {\bf{ }} \\%
   {\BBlue{13}} &   &   \\%
   {\BBlue{5}} &   &   \\%
   {\BRed{5}} & {\bf{ }} & {\bf{ }} \\%
   {\BBlue{ }} &   &   \\%
   \multicolumn{3}{@{\hspace{1pt}}c}{ }\\%
   \end{tabular*}
   \vfill}}
  \end{minipage}}}

%%  Carbon
\newcommand{\ElemC}{{%
  \begin{minipage}{67mm}%
  \vspace{1mm}

  {\Huge{\hspace{1mm} {\textbf{C}} \hfill \hfil{\textbf{6}} \hspace{1mm}}} %

  \vspace{6mm}

  {\Huge{\hfill {\Name{carbon}} \hfill}}

  \vspace{6mm}

  {\Large{
  \begin{tabular*}{67mm}%
   {@{\hspace{5pt}}{r}@{\extracolsep{\fill}}r@{\extracolsep{\fill}}r}%
   %\multicolumn{3}{@{\hspace{1pt}}c}{ }\\%
   %\multicolumn{3}{c}{{\Huge{\BBlue{carbon}}} }\\%
   % \multicolumn{3}{@{\hspace{1pt}}c}{ }\\%
   {\BRed{284}}  & {\bf{277}} &  {\bf{ }} \\%
   {\BBlue{18}} &   &   \\%
   {\BBlue{7}} &   &   \\%
   {\BRed{7}} & {\bf{ }} & {\bf{ }} \\%
   {\BBlue{ }} &   &   \\%
   \multicolumn{3}{@{\hspace{1pt}}c}{ }\\%
   \end{tabular*}
   \vfill}}
  \end{minipage}}}

%%  Nitrogen
\newcommand{\ElemN}{{%
  \begin{minipage}{67mm}%
  \vspace{1mm}

  {\Huge{\hspace{1mm} {\textbf{N}} \hfill \hfil{\textbf{7}} \hspace{1mm}}} %

  \vspace{6mm}

  {\Huge{\hfill {\Name{nitrogen}} \hfill}}

  \vspace{6mm}

  {\Large{
  \begin{tabular*}{67mm}%
   {@{\hspace{5pt}}{r}@{\extracolsep{\fill}}r@{\extracolsep{\fill}}r}%
   %\multicolumn{3}{@{\hspace{1pt}}c}{ }\\%
   %\multicolumn{3}{c}{{\Huge{\BBlue{nitrogen}}} }\\%
   % \multicolumn{3}{@{\hspace{1pt}}c}{ }\\%
   {\BRed{410}}  & {\bf{392}} &  {\bf{ }} \\%
   {\BBlue{37}} &   &   \\%
   {\BBlue{18}} &   &   \\%
   {\BRed{18}} & {\bf{ }} & {\bf{ }} \\%
   {\BBlue{ }} &   &   \\%
   \multicolumn{3}{@{\hspace{1pt}}c}{ }\\%
   \end{tabular*}
   \vfill}}
  \end{minipage}}}

%%  Oxygen
\newcommand{\ElemO}{{%
  \begin{minipage}{67mm}%
  \vspace{1mm}

  {\Huge{\hspace{1mm} {\textbf{O}} \hfill \hfil{\textbf{8}} \hspace{1mm}}} %

  \vspace{6mm}

  {\Huge{\hfill {\Name{oxygen}} \hfill}}

  \vspace{6mm}

  {\Large{
  \begin{tabular*}{67mm}%
   {@{\hspace{5pt}}{r}@{\extracolsep{\fill}}r@{\extracolsep{\fill}}r}%
   %\multicolumn{3}{@{\hspace{1pt}}c}{ }\\%
   %\multicolumn{3}{c}{{\Huge{\BBlue{oxygen}}} }\\%
   % \multicolumn{3}{@{\hspace{1pt}}c}{ }\\%
   {\BRed{543}}  & {\bf{525}} &  {\bf{ }} \\%
   {\BBlue{42}} &   &   \\%
   {\BBlue{18}} &   &   \\%
   {\BRed{18}} & {\bf{ }} & {\bf{ }} \\%
   {\BBlue{ }} &   &   \\%
   \multicolumn{3}{@{\hspace{1pt}}c}{ }\\%
   \end{tabular*}
   \vfill}}
  \end{minipage}}}

%%  Fluorine
\newcommand{\ElemF}{{%
  \begin{minipage}{67mm}%
  \vspace{1mm}

  {\Huge{\hspace{1mm} {\textbf{F}} \hfill \hfil{\textbf{9}} \hspace{1mm}}} %

  \vspace{6mm}

  {\Huge{\hfill {\Name{fluorine}} \hfill}}

  \vspace{6mm}

  {\Large{
  \begin{tabular*}{67mm}%
   {@{\hspace{5pt}}{r}@{\extracolsep{\fill}}r@{\extracolsep{\fill}}r}%
   %\multicolumn{3}{@{\hspace{1pt}}c}{ }\\%
   %\multicolumn{3}{c}{{\Huge{\BBlue{fluorine}}} }\\%
   % \multicolumn{3}{@{\hspace{1pt}}c}{ }\\%
   {\BRed{697}}  & {\bf{677}} &  {\bf{ }} \\%
   {\BBlue{45}} &   &   \\%
   {\BBlue{20}} &   &   \\%
   {\BRed{20}} & {\bf{ }} & {\bf{ }} \\%
   {\BBlue{ }} &   &   \\%
   \multicolumn{3}{@{\hspace{1pt}}c}{ }\\%
   \end{tabular*}
   \vfill}}
  \end{minipage}}}

%%  Neon
\newcommand{\ElemNe}{{%
  \begin{minipage}{67mm}%
  \vspace{1mm}

  {\Huge{\hspace{1mm} {\textbf{Ne}} \hfill \hfil{\textbf{10}} \hspace{1mm}}} %

  \vspace{6mm}

  {\Huge{\hfill {\Name{neon}} \hfill}}

  \vspace{6mm}

  {\Large{
  \begin{tabular*}{67mm}%
   {@{\hspace{5pt}}{r}@{\extracolsep{\fill}}r@{\extracolsep{\fill}}r}%
   %\multicolumn{3}{@{\hspace{1pt}}c}{ }\\%
   %\multicolumn{3}{c}{{\Huge{\BBlue{neon}}} }\\%
   % \multicolumn{3}{@{\hspace{1pt}}c}{ }\\%
   {\BRed{870}}  & {\bf{849}} &  {\bf{ }} \\%
   {\BBlue{49}} &   &   \\%
   {\BBlue{22}} &   &   \\%
   {\BRed{22}} & {\bf{ }} & {\bf{ }} \\%
   {\BBlue{ }} &   &   \\%
   \multicolumn{3}{@{\hspace{1pt}}c}{ }\\%
   \end{tabular*}
   \vfill}}
  \end{minipage}}}

%%  Sodium
\newcommand{\ElemNa}{{%
  \begin{minipage}{67mm}%
  \vspace{1mm}

  {\Huge{\hspace{1mm} {\textbf{Na}} \hfill \hfil{\textbf{11}} \hspace{1mm}}} %

  \vspace{6mm}

  {\Huge{\hfill {\Name{sodium}} \hfill}}

  \vspace{6mm}

  {\Large{
  \begin{tabular*}{67mm}%
   {@{\hspace{5pt}}{r}@{\extracolsep{\fill}}r@{\extracolsep{\fill}}r}%
   %\multicolumn{3}{@{\hspace{1pt}}c}{ }\\%
   %\multicolumn{3}{c}{{\Huge{\BBlue{sodium}}} }\\%
   % \multicolumn{3}{@{\hspace{1pt}}c}{ }\\%
   {\BRed{1071}}  & {\bf{1040}} &  {\bf{ }} \\%
   {\BBlue{64}} &   &   \\%
   {\BBlue{30}} &   &   \\%
   {\BRed{31}} & {\bf{ }} & {\bf{ }} \\%
   {\BBlue{ }} &   &   \\%
   \multicolumn{3}{@{\hspace{1pt}}c}{ }\\%
   \end{tabular*}
   \vfill}}
  \end{minipage}}}

%%  Magnesium
\newcommand{\ElemMg}{{%
  \begin{minipage}{67mm}%
  \vspace{1mm}

  {\Huge{\hspace{1mm} {\textbf{Mg}} \hfill \hfil{\textbf{12}} \hspace{1mm}}} %

  \vspace{6mm}

  {\Huge{\hfill {\Name{magnesium}} \hfill}}

  \vspace{6mm}

  {\Large{
  \begin{tabular*}{67mm}%
   {@{\hspace{5pt}}{r}@{\extracolsep{\fill}}r@{\extracolsep{\fill}}r}%
   %\multicolumn{3}{@{\hspace{1pt}}c}{ }\\%
   %\multicolumn{3}{c}{{\Huge{\BBlue{magnesium}}} }\\%
   % \multicolumn{3}{@{\hspace{1pt}}c}{ }\\%
   {\BRed{1303}}  & {\bf{1254}} &  {\bf{1302}} \\%
   {\BBlue{89}} & 88 & 88 \\%
   {\BBlue{50}} &   &   \\%
   {\BRed{49}} & {\bf{ }} & {\bf{ }} \\%
   {\BBlue{ }} &   &   \\%
   \multicolumn{3}{@{\hspace{1pt}}c}{ }\\%
   \end{tabular*}
   \vfill}}
  \end{minipage}}}

%%  Aluminum
\newcommand{\ElemAl}{{%
  \begin{minipage}{67mm}%
  \vspace{1mm}

  {\Huge{\hspace{1mm} {\textbf{Al}} \hfill \hfil{\textbf{13}} \hspace{1mm}}} %

  \vspace{6mm}

  {\Huge{\hfill {\Name{aluminum}} \hfill}}

  \vspace{6mm}

  {\Large{
  \begin{tabular*}{67mm}%
   {@{\hspace{5pt}}{r}@{\extracolsep{\fill}}r@{\extracolsep{\fill}}r}%
   %\multicolumn{3}{@{\hspace{1pt}}c}{ }\\%
   %\multicolumn{3}{c}{{\Huge{\BBlue{aluminum}}} }\\%
   % \multicolumn{3}{@{\hspace{1pt}}c}{ }\\%
   {\BRed{1559}}  & {\bf{1487}} &  {\bf{1557}} \\%
   {\BBlue{118}} & 116 & 116 \\%
   {\BBlue{73}} &   &   \\%
   {\BRed{73}} & {\bf{ }} & {\bf{ }} \\%
   {\BBlue{ }} &   &   \\%
   \multicolumn{3}{@{\hspace{1pt}}c}{ }\\%
   \end{tabular*}
   \vfill}}
  \end{minipage}}}

%%  Silicon
\newcommand{\ElemSi}{{%
  \begin{minipage}{67mm}%
  \vspace{1mm}

  {\Huge{\hspace{1mm} {\textbf{Si}} \hfill \hfil{\textbf{14}} \hspace{1mm}}} %

  \vspace{6mm}

  {\Huge{\hfill {\Name{silicon}} \hfill}}

  \vspace{6mm}

  {\Large{
  \begin{tabular*}{67mm}%
   {@{\hspace{5pt}}{r}@{\extracolsep{\fill}}r@{\extracolsep{\fill}}r}%
   %\multicolumn{3}{@{\hspace{1pt}}c}{ }\\%
   %\multicolumn{3}{c}{{\Huge{\BBlue{silicon}}} }\\%
   % \multicolumn{3}{@{\hspace{1pt}}c}{ }\\%
   {\BRed{1839}}  & {\bf{1740}} &  {\bf{1837}} \\%
   {\BBlue{150}} & 148 & 148 \\%
   {\BBlue{100}} &   &   \\%
   {\BRed{99}} & {\bf{ }} & {\bf{ }} \\%
   {\BBlue{ }} &   &   \\%
   \multicolumn{3}{@{\hspace{1pt}}c}{ }\\%
   \end{tabular*}
   \vfill}}
  \end{minipage}}}

%%  Phosphorus
\newcommand{\ElemP}{{%
  \begin{minipage}{67mm}%
  \vspace{1mm}

  {\Huge{\hspace{1mm} {\textbf{P}} \hfill \hfil{\textbf{15}} \hspace{1mm}}} %

  \vspace{6mm}

  {\Huge{\hfill {\Name{phosphorus}} \hfill}}

  \vspace{6mm}

  {\Large{
  \begin{tabular*}{67mm}%
   {@{\hspace{5pt}}{r}@{\extracolsep{\fill}}r@{\extracolsep{\fill}}r}%
   %\multicolumn{3}{@{\hspace{1pt}}c}{ }\\%
   %\multicolumn{3}{c}{{\Huge{\BBlue{phosphorus}}} }\\%
   % \multicolumn{3}{@{\hspace{1pt}}c}{ }\\%
   {\BRed{2146}}  & {\bf{2011}} &  {\bf{2140}} \\%
   {\BBlue{189}} & 183 & 182 \\%
   {\BBlue{136}} &   &   \\%
   {\BRed{135}} & {\bf{ }} & {\bf{ }} \\%
   {\BBlue{ }} &   &   \\%
   \multicolumn{3}{@{\hspace{1pt}}c}{ }\\%
   \end{tabular*}
   \vfill}}
  \end{minipage}}}

%%  Sulfur
\newcommand{\ElemS}{{%
  \begin{minipage}{67mm}%
  \vspace{1mm}

  {\Huge{\hspace{1mm} {\textbf{S}} \hfill \hfil{\textbf{16}} \hspace{1mm}}} %

  \vspace{6mm}

  {\Huge{\hfill {\Name{sulfur}} \hfill}}

  \vspace{6mm}

  {\Large{
  \begin{tabular*}{67mm}%
   {@{\hspace{5pt}}{r}@{\extracolsep{\fill}}r@{\extracolsep{\fill}}r}%
   %\multicolumn{3}{@{\hspace{1pt}}c}{ }\\%
   %\multicolumn{3}{c}{{\Huge{\BBlue{sulfur}}} }\\%
   % \multicolumn{3}{@{\hspace{1pt}}c}{ }\\%
   {\BRed{2472}}  & {\bf{2310}} &  {\bf{2465}} \\%
   {\BBlue{231}} & 224 & 223 \\%
   {\BBlue{164}} &   &   \\%
   {\BRed{163}} & {\bf{ }} & {\bf{ }} \\%
   {\BBlue{ }} &   &   \\%
   \multicolumn{3}{@{\hspace{1pt}}c}{ }\\%
   \end{tabular*}
   \vfill}}
  \end{minipage}}}

%%  Chlorine
\newcommand{\ElemCl}{{%
  \begin{minipage}{67mm}%
  \vspace{1mm}

  {\Huge{\hspace{1mm} {\textbf{Cl}} \hfill \hfil{\textbf{17}} \hspace{1mm}}} %

  \vspace{6mm}

  {\Huge{\hfill {\Name{chlorine}} \hfill}}

  \vspace{6mm}

  {\Large{
  \begin{tabular*}{67mm}%
   {@{\hspace{5pt}}{r}@{\extracolsep{\fill}}r@{\extracolsep{\fill}}r}%
   %\multicolumn{3}{@{\hspace{1pt}}c}{ }\\%
   %\multicolumn{3}{c}{{\Huge{\BBlue{chlorine}}} }\\%
   % \multicolumn{3}{@{\hspace{1pt}}c}{ }\\%
   {\BRed{2822}}  & {\bf{2622}} &  {\bf{2812}} \\%
   {\BBlue{270}} & 260 & 260 \\%
   {\BBlue{202}} &   &   \\%
   {\BRed{200}} & {\bf{ }} & {\bf{ }} \\%
   {\BBlue{ }} &   &   \\%
   \multicolumn{3}{@{\hspace{1pt}}c}{ }\\%
   \end{tabular*}
   \vfill}}
  \end{minipage}}}

%%  Argon
\newcommand{\ElemAr}{{%
  \begin{minipage}{67mm}%
  \vspace{1mm}

  {\Huge{\hspace{1mm} {\textbf{Ar}} \hfill \hfil{\textbf{18}} \hspace{1mm}}} %

  \vspace{6mm}

  {\Huge{\hfill {\Name{argon}} \hfill}}

  \vspace{6mm}

  {\Large{
  \begin{tabular*}{67mm}%
   {@{\hspace{5pt}}{r}@{\extracolsep{\fill}}r@{\extracolsep{\fill}}r}%
   %\multicolumn{3}{@{\hspace{1pt}}c}{ }\\%
   %\multicolumn{3}{c}{{\Huge{\BBlue{argon}}} }\\%
   % \multicolumn{3}{@{\hspace{1pt}}c}{ }\\%
   {\BRed{3206}}  & {\bf{2958}} &  {\bf{3190}} \\%
   {\BBlue{326}} & 311 & 310 \\%
   {\BBlue{251}} &   &   \\%
   {\BRed{248}} & {\bf{ }} & {\bf{ }} \\%
   {\BBlue{ }} &   &   \\%
   \multicolumn{3}{@{\hspace{1pt}}c}{ }\\%
   \end{tabular*}
   \vfill}}
  \end{minipage}}}

%%  Potassium
\newcommand{\ElemK}{{%
  \begin{minipage}{67mm}%
  \vspace{1mm}

  {\Huge{\hspace{1mm} {\textbf{K}} \hfill \hfil{\textbf{19}} \hspace{1mm}}} %

  \vspace{6mm}

  {\Huge{\hfill {\Name{potassium}} \hfill}}

  \vspace{6mm}

  {\Large{
  \begin{tabular*}{67mm}%
   {@{\hspace{5pt}}{r}@{\extracolsep{\fill}}r@{\extracolsep{\fill}}r}%
   %\multicolumn{3}{@{\hspace{1pt}}c}{ }\\%
   %\multicolumn{3}{c}{{\Huge{\BBlue{potassium}}} }\\%
   % \multicolumn{3}{@{\hspace{1pt}}c}{ }\\%
   {\BRed{3608}}  & {\bf{3314}} &  {\bf{3590}} \\%
   {\BBlue{379}} & 360 & 360 \\%
   {\BBlue{297}} &   &   \\%
   {\BRed{295}} & {\bf{ }} & {\bf{ }} \\%
   {\BBlue{ }} &   &   \\%
   \multicolumn{3}{@{\hspace{1pt}}c}{ }\\%
   \end{tabular*}
   \vfill}}
  \end{minipage}}}

%%  Calcium
\newcommand{\ElemCa}{{%
  \begin{minipage}{67mm}%
  \vspace{1mm}

  {\Huge{\hspace{1mm} {\textbf{Ca}} \hfill \hfil{\textbf{20}} \hspace{1mm}}} %

  \vspace{6mm}

  {\Huge{\hfill {\Name{calcium}} \hfill}}

  \vspace{6mm}

  {\Large{
  \begin{tabular*}{67mm}%
   {@{\hspace{5pt}}{r}@{\extracolsep{\fill}}r@{\extracolsep{\fill}}r}%
   %\multicolumn{3}{@{\hspace{1pt}}c}{ }\\%
   %\multicolumn{3}{c}{{\Huge{\BBlue{calcium}}} }\\%
   % \multicolumn{3}{@{\hspace{1pt}}c}{ }\\%
   {\BRed{4039}}  & {\bf{3692}} &  {\bf{4013}} \\%
   {\BBlue{438}} & 413 & 413 \\%
   {\BBlue{350}} &   &   \\%
   {\BRed{346}} & {\bf{ }} & {\bf{ }} \\%
   {\BBlue{ }} &   &   \\%
   \multicolumn{3}{@{\hspace{1pt}}c}{ }\\%
   \end{tabular*}
   \vfill}}
  \end{minipage}}}

%%  Scandium
\newcommand{\ElemSc}{{%
  \begin{minipage}{67mm}%
  \vspace{1mm}

  {\Huge{\hspace{1mm} {\textbf{Sc}} \hfill \hfil{\textbf{21}} \hspace{1mm}}} %

  \vspace{6mm}

  {\Huge{\hfill {\Name{scandium}} \hfill}}

  \vspace{6mm}

  {\Large{
  \begin{tabular*}{67mm}%
   {@{\hspace{5pt}}{r}@{\extracolsep{\fill}}r@{\extracolsep{\fill}}r}%
   %\multicolumn{3}{@{\hspace{1pt}}c}{ }\\%
   %\multicolumn{3}{c}{{\Huge{\BBlue{scandium}}} }\\%
   % \multicolumn{3}{@{\hspace{1pt}}c}{ }\\%
   {\BRed{4492}}  & {\bf{4093}} &  {\bf{4464}} \\%
   {\BBlue{498}} & 470 & 470 \\%
   {\BBlue{404}} &   &   \\%
   {\BRed{399}} & {\bf{ }} & {\bf{ }} \\%
   {\BBlue{ }} &   &   \\%
   \multicolumn{3}{@{\hspace{1pt}}c}{ }\\%
   \end{tabular*}
   \vfill}}
  \end{minipage}}}

%%  Titanium
\newcommand{\ElemTi}{{%
  \begin{minipage}{67mm}%
  \vspace{1mm}

  {\Huge{\hspace{1mm} {\textbf{Ti}} \hfill \hfil{\textbf{22}} \hspace{1mm}}} %

  \vspace{6mm}

  {\Huge{\hfill {\Name{titanium}} \hfill}}

  \vspace{6mm}

  {\Large{
  \begin{tabular*}{67mm}%
   {@{\hspace{5pt}}{r}@{\extracolsep{\fill}}r@{\extracolsep{\fill}}r}%
   %\multicolumn{3}{@{\hspace{1pt}}c}{ }\\%
   %\multicolumn{3}{c}{{\Huge{\BBlue{titanium}}} }\\%
   % \multicolumn{3}{@{\hspace{1pt}}c}{ }\\%
   {\BRed{4966}}  & {\bf{4512}} &  {\bf{4933}} \\%
   {\BBlue{561}} & 528 & 528 \\%
   {\BBlue{460}} & 458 &   \\%
   {\BRed{454}} & {\bf{452}} & {\bf{ }} \\%
   {\BBlue{2}} &   &   \\%
   \multicolumn{3}{@{\hspace{1pt}}c}{ }\\%
   \end{tabular*}
   \vfill}}
  \end{minipage}}}

%%  Vanadium
\newcommand{\ElemV}{{%
  \begin{minipage}{67mm}%
  \vspace{1mm}

  {\Huge{\hspace{1mm} {\textbf{V}} \hfill \hfil{\textbf{23}} \hspace{1mm}}} %

  \vspace{6mm}

  {\Huge{\hfill {\Name{vanadium}} \hfill}}

  \vspace{6mm}

  {\Large{
  \begin{tabular*}{67mm}%
   {@{\hspace{5pt}}{r}@{\extracolsep{\fill}}r@{\extracolsep{\fill}}r}%
   %\multicolumn{3}{@{\hspace{1pt}}c}{ }\\%
   %\multicolumn{3}{c}{{\Huge{\BBlue{vanadium}}} }\\%
   % \multicolumn{3}{@{\hspace{1pt}}c}{ }\\%
   {\BRed{5465}}  & {\bf{4953}} &  {\bf{5428}} \\%
   {\BBlue{627}} & 590 & 590 \\%
   {\BBlue{520}} & 518 &   \\%
   {\BRed{512}} & {\bf{510}} & {\bf{ }} \\%
   {\BBlue{2}} &   &   \\%
   \multicolumn{3}{@{\hspace{1pt}}c}{ }\\%
   \end{tabular*}
   \vfill}}
  \end{minipage}}}

%%  Chromium
\newcommand{\ElemCr}{{%
  \begin{minipage}{67mm}%
  \vspace{1mm}

  {\Huge{\hspace{1mm} {\textbf{Cr}} \hfill \hfil{\textbf{24}} \hspace{1mm}}} %

  \vspace{6mm}

  {\Huge{\hfill {\Name{chromium}} \hfill}}

  \vspace{6mm}

  {\Large{
  \begin{tabular*}{67mm}%
   {@{\hspace{5pt}}{r}@{\extracolsep{\fill}}r@{\extracolsep{\fill}}r}%
   %\multicolumn{3}{@{\hspace{1pt}}c}{ }\\%
   %\multicolumn{3}{c}{{\Huge{\BBlue{chromium}}} }\\%
   % \multicolumn{3}{@{\hspace{1pt}}c}{ }\\%
   {\BRed{5989}}  & {\bf{5415}} &  {\bf{5947}} \\%
   {\BBlue{696}} & 654 & 654 \\%
   {\BBlue{584}} & 582 &   \\%
   {\BRed{574}} & {\bf{572}} & {\bf{ }} \\%
   {\BBlue{2}} &   &   \\%
   \multicolumn{3}{@{\hspace{1pt}}c}{ }\\%
   \end{tabular*}
   \vfill}}
  \end{minipage}}}

%%  Manganese
\newcommand{\ElemMn}{{%
  \begin{minipage}{67mm}%
  \vspace{1mm}

  {\Huge{\hspace{1mm} {\textbf{Mn}} \hfill \hfil{\textbf{25}} \hspace{1mm}}} %

  \vspace{6mm}

  {\Huge{\hfill {\Name{manganese}} \hfill}}

  \vspace{6mm}

  {\Large{
  \begin{tabular*}{67mm}%
   {@{\hspace{5pt}}{r}@{\extracolsep{\fill}}r@{\extracolsep{\fill}}r}%
   %\multicolumn{3}{@{\hspace{1pt}}c}{ }\\%
   %\multicolumn{3}{c}{{\Huge{\BBlue{manganese}}} }\\%
   % \multicolumn{3}{@{\hspace{1pt}}c}{ }\\%
   {\BRed{6539}}  & {\bf{5900}} &  {\bf{6492}} \\%
   {\BBlue{769}} & 722 & 722 \\%
   {\BBlue{650}} & 648 &   \\%
   {\BRed{639}} & {\bf{637}} & {\bf{ }} \\%
   {\BBlue{2}} &   &   \\%
   \multicolumn{3}{@{\hspace{1pt}}c}{ }\\%
   \end{tabular*}
   \vfill}}
  \end{minipage}}}

%%  Iron
\newcommand{\ElemFe}{{%
  \begin{minipage}{67mm}%
  \vspace{1mm}

  {\Huge{\hspace{1mm} {\textbf{Fe}} \hfill \hfil{\textbf{26}} \hspace{1mm}}} %

  \vspace{6mm}

  {\Huge{\hfill {\Name{iron}} \hfill}}

  \vspace{6mm}

  {\Large{
  \begin{tabular*}{67mm}%
   {@{\hspace{5pt}}{r}@{\extracolsep{\fill}}r@{\extracolsep{\fill}}r}%
   %\multicolumn{3}{@{\hspace{1pt}}c}{ }\\%
   %\multicolumn{3}{c}{{\Huge{\BBlue{iron}}} }\\%
   % \multicolumn{3}{@{\hspace{1pt}}c}{ }\\%
   {\BRed{7112}}  & {\bf{6405}} &  {\bf{7059}} \\%
   {\BBlue{845}} & 792 & 792 \\%
   {\BBlue{720}} & 718 &   \\%
   {\BRed{707}} & {\bf{705}} & {\bf{ }} \\%
   {\BBlue{2}} &   &   \\%
   \multicolumn{3}{@{\hspace{1pt}}c}{ }\\%
   \end{tabular*}
   \vfill}}
  \end{minipage}}}

%%  Cobalt
\newcommand{\ElemCo}{{%
  \begin{minipage}{67mm}%
  \vspace{1mm}

  {\Huge{\hspace{1mm} {\textbf{Co}} \hfill \hfil{\textbf{27}} \hspace{1mm}}} %

  \vspace{6mm}

  {\Huge{\hfill {\Name{cobalt}} \hfill}}

  \vspace{6mm}

  {\Large{
  \begin{tabular*}{67mm}%
   {@{\hspace{5pt}}{r}@{\extracolsep{\fill}}r@{\extracolsep{\fill}}r}%
   %\multicolumn{3}{@{\hspace{1pt}}c}{ }\\%
   %\multicolumn{3}{c}{{\Huge{\BBlue{cobalt}}} }\\%
   % \multicolumn{3}{@{\hspace{1pt}}c}{ }\\%
   {\BRed{7709}}  & {\bf{6931}} &  {\bf{7649}} \\%
   {\BBlue{925}} & 865 & 866 \\%
   {\BBlue{793}} & 790 &   \\%
   {\BRed{778}} & {\bf{775}} & {\bf{ }} \\%
   {\BBlue{3}} &   &   \\%
   \multicolumn{3}{@{\hspace{1pt}}c}{ }\\%
   \end{tabular*}
   \vfill}}
  \end{minipage}}}

%%  Nickel
\newcommand{\ElemNi}{{%
  \begin{minipage}{67mm}%
  \vspace{1mm}

  {\Huge{\hspace{1mm} {\textbf{Ni}} \hfill \hfil{\textbf{28}} \hspace{1mm}}} %

  \vspace{6mm}

  {\Huge{\hfill {\Name{nickel}} \hfill}}

  \vspace{6mm}

  {\Large{
  \begin{tabular*}{67mm}%
   {@{\hspace{5pt}}{r}@{\extracolsep{\fill}}r@{\extracolsep{\fill}}r}%
   %\multicolumn{3}{@{\hspace{1pt}}c}{ }\\%
   %\multicolumn{3}{c}{{\Huge{\BBlue{nickel}}} }\\%
   % \multicolumn{3}{@{\hspace{1pt}}c}{ }\\%
   {\BRed{8333}}  & {\bf{7480}} &  {\bf{8267}} \\%
   {\BBlue{1009}} & 942 & 941 \\%
   {\BBlue{870}} & 866 &   \\%
   {\BRed{853}} & {\bf{849}} & {\bf{ }} \\%
   {\BBlue{4}} &   &   \\%
   \multicolumn{3}{@{\hspace{1pt}}c}{ }\\%
   \end{tabular*}
   \vfill}}
  \end{minipage}}}

%%  Copper
\newcommand{\ElemCu}{{%
  \begin{minipage}{67mm}%
  \vspace{1mm}

  {\Huge{\hspace{1mm} {\textbf{Cu}} \hfill \hfil{\textbf{29}} \hspace{1mm}}} %

  \vspace{6mm}

  {\Huge{\hfill {\Name{copper}} \hfill}}

  \vspace{6mm}

  {\Large{
  \begin{tabular*}{67mm}%
   {@{\hspace{5pt}}{r}@{\extracolsep{\fill}}r@{\extracolsep{\fill}}r}%
   %\multicolumn{3}{@{\hspace{1pt}}c}{ }\\%
   %\multicolumn{3}{c}{{\Huge{\BBlue{copper}}} }\\%
   % \multicolumn{3}{@{\hspace{1pt}}c}{ }\\%
   {\BRed{8979}}  & {\bf{8046}} &  {\bf{8904}} \\%
   {\BBlue{1097}} & 1022 & 1019 \\%
   {\BBlue{952}} & 947 &   \\%
   {\BRed{933}} & {\bf{928}} & {\bf{ }} \\%
   {\BBlue{5}} &   &   \\%
   \multicolumn{3}{@{\hspace{1pt}}c}{ }\\%
   \end{tabular*}
   \vfill}}
  \end{minipage}}}

%%  Zinc
\newcommand{\ElemZn}{{%
  \begin{minipage}{67mm}%
  \vspace{1mm}

  {\Huge{\hspace{1mm} {\textbf{Zn}} \hfill \hfil{\textbf{30}} \hspace{1mm}}} %

  \vspace{6mm}

  {\Huge{\hfill {\Name{zinc}} \hfill}}

  \vspace{6mm}

  {\Large{
  \begin{tabular*}{67mm}%
   {@{\hspace{5pt}}{r}@{\extracolsep{\fill}}r@{\extracolsep{\fill}}r}%
   %\multicolumn{3}{@{\hspace{1pt}}c}{ }\\%
   %\multicolumn{3}{c}{{\Huge{\BBlue{zinc}}} }\\%
   % \multicolumn{3}{@{\hspace{1pt}}c}{ }\\%
   {\BRed{9659}}  & {\bf{8637}} &  {\bf{9570}} \\%
   {\BBlue{1196}} & 1108 & 1105 \\%
   {\BBlue{1045}} & 1035 &   \\%
   {\BRed{1022}} & {\bf{1012}} & {\bf{ }} \\%
   {\BBlue{10}} &   &   \\%
   \multicolumn{3}{@{\hspace{1pt}}c}{ }\\%
   \end{tabular*}
   \vfill}}
  \end{minipage}}}

%%  Gallium
\newcommand{\ElemGa}{{%
  \begin{minipage}{67mm}%
  \vspace{1mm}

  {\Huge{\hspace{1mm} {\textbf{Ga}} \hfill \hfil{\textbf{31}} \hspace{1mm}}} %

  \vspace{6mm}

  {\Huge{\hfill {\Name{gallium}} \hfill}}

  \vspace{6mm}

  {\Large{
  \begin{tabular*}{67mm}%
   {@{\hspace{5pt}}{r}@{\extracolsep{\fill}}r@{\extracolsep{\fill}}r}%
   %\multicolumn{3}{@{\hspace{1pt}}c}{ }\\%
   %\multicolumn{3}{c}{{\Huge{\BBlue{gallium}}} }\\%
   % \multicolumn{3}{@{\hspace{1pt}}c}{ }\\%
   {\BRed{10367}}  & {\bf{9251}} &  {\bf{10267}} \\%
   {\BBlue{1299}} & 1199 & 1196 \\%
   {\BBlue{1143}} & 1125 &   \\%
   {\BRed{1116}} & {\bf{1098}} & {\bf{ }} \\%
   {\BBlue{19}} &   &   \\%
   \multicolumn{3}{@{\hspace{1pt}}c}{ }\\%
   \end{tabular*}
   \vfill}}
  \end{minipage}}}

%%  Germanium
\newcommand{\ElemGe}{{%
  \begin{minipage}{67mm}%
  \vspace{1mm}

  {\Huge{\hspace{1mm} {\textbf{Ge}} \hfill \hfil{\textbf{32}} \hspace{1mm}}} %

  \vspace{6mm}

  {\Huge{\hfill {\Name{germanium}} \hfill}}

  \vspace{6mm}

  {\Large{
  \begin{tabular*}{67mm}%
   {@{\hspace{5pt}}{r}@{\extracolsep{\fill}}r@{\extracolsep{\fill}}r}%
   %\multicolumn{3}{@{\hspace{1pt}}c}{ }\\%
   %\multicolumn{3}{c}{{\Huge{\BBlue{germanium}}} }\\%
   % \multicolumn{3}{@{\hspace{1pt}}c}{ }\\%
   {\BRed{11103}}  & {\bf{9886}} &  {\bf{10982}} \\%
   {\BBlue{1415}} & 1294 & 1290 \\%
   {\BBlue{1248}} & 1218 &   \\%
   {\BRed{1217}} & {\bf{1188}} & {\bf{ }} \\%
   {\BBlue{29}} &   &   \\%
   \multicolumn{3}{@{\hspace{1pt}}c}{ }\\%
   \end{tabular*}
   \vfill}}
  \end{minipage}}}

%%  Arsenic
\newcommand{\ElemAs}{{%
  \begin{minipage}{67mm}%
  \vspace{1mm}

  {\Huge{\hspace{1mm} {\textbf{As}} \hfill \hfil{\textbf{33}} \hspace{1mm}}} %

  \vspace{6mm}

  {\Huge{\hfill {\Name{arsenic}} \hfill}}

  \vspace{6mm}

  {\Large{
  \begin{tabular*}{67mm}%
   {@{\hspace{5pt}}{r}@{\extracolsep{\fill}}r@{\extracolsep{\fill}}r}%
   %\multicolumn{3}{@{\hspace{1pt}}c}{ }\\%
   %\multicolumn{3}{c}{{\Huge{\BBlue{arsenic}}} }\\%
   % \multicolumn{3}{@{\hspace{1pt}}c}{ }\\%
   {\BRed{11867}}  & {\bf{10543}} &  {\bf{11726}} \\%
   {\BBlue{1527}} & 1386 & 1381 \\%
   {\BBlue{1359}} & 1317 &   \\%
   {\BRed{1324}} & {\bf{1282}} & {\bf{ }} \\%
   {\BBlue{42}} &   &   \\%
   \multicolumn{3}{@{\hspace{1pt}}c}{ }\\%
   \end{tabular*}
   \vfill}}
  \end{minipage}}}

%%  Selenium
\newcommand{\ElemSe}{{%
  \begin{minipage}{67mm}%
  \vspace{1mm}

  {\Huge{\hspace{1mm} {\textbf{Se}} \hfill \hfil{\textbf{34}} \hspace{1mm}}} %

  \vspace{6mm}

  {\Huge{\hfill {\Name{selenium}} \hfill}}

  \vspace{6mm}

  {\Large{
  \begin{tabular*}{67mm}%
   {@{\hspace{5pt}}{r}@{\extracolsep{\fill}}r@{\extracolsep{\fill}}r}%
   %\multicolumn{3}{@{\hspace{1pt}}c}{ }\\%
   %\multicolumn{3}{c}{{\Huge{\BBlue{selenium}}} }\\%
   % \multicolumn{3}{@{\hspace{1pt}}c}{ }\\%
   {\BRed{12658}}  & {\bf{11224}} &  {\bf{12497}} \\%
   {\BBlue{1652}} & 1491 & 1486 \\%
   {\BBlue{1474}} & 1419 &   \\%
   {\BRed{1434}} & {\bf{1379}} & {\bf{ }} \\%
   {\BBlue{55}} &   &   \\%
   \multicolumn{3}{@{\hspace{1pt}}c}{ }\\%
   \end{tabular*}
   \vfill}}
  \end{minipage}}}

%%  Bromine
\newcommand{\ElemBr}{{%
  \begin{minipage}{67mm}%
  \vspace{1mm}

  {\Huge{\hspace{1mm} {\textbf{Br}} \hfill \hfil{\textbf{35}} \hspace{1mm}}} %

  \vspace{6mm}

  {\Huge{\hfill {\Name{bromine}} \hfill}}

  \vspace{6mm}

  {\Large{
  \begin{tabular*}{67mm}%
   {@{\hspace{5pt}}{r}@{\extracolsep{\fill}}r@{\extracolsep{\fill}}r}%
   %\multicolumn{3}{@{\hspace{1pt}}c}{ }\\%
   %\multicolumn{3}{c}{{\Huge{\BBlue{bromine}}} }\\%
   % \multicolumn{3}{@{\hspace{1pt}}c}{ }\\%
   {\BRed{13474}}  & {\bf{11924}} &  {\bf{13292}} \\%
   {\BBlue{1782}} & 1600 & 1593 \\%
   {\BBlue{1596}} & 1526 &   \\%
   {\BRed{1550}} & {\bf{1481}} & {\bf{ }} \\%
   {\BBlue{69}} &   &   \\%
   \multicolumn{3}{@{\hspace{1pt}}c}{ }\\%
   \end{tabular*}
   \vfill}}
  \end{minipage}}}

%%  Krypton
\newcommand{\ElemKr}{{%
  \begin{minipage}{67mm}%
  \vspace{1mm}

  {\Huge{\hspace{1mm} {\textbf{Kr}} \hfill \hfil{\textbf{36}} \hspace{1mm}}} %

  \vspace{6mm}

  {\Huge{\hfill {\Name{krypton}} \hfill}}

  \vspace{6mm}

  {\Large{
  \begin{tabular*}{67mm}%
   {@{\hspace{5pt}}{r}@{\extracolsep{\fill}}r@{\extracolsep{\fill}}r}%
   %\multicolumn{3}{@{\hspace{1pt}}c}{ }\\%
   %\multicolumn{3}{c}{{\Huge{\BBlue{krypton}}} }\\%
   % \multicolumn{3}{@{\hspace{1pt}}c}{ }\\%
   {\BRed{14326}}  & {\bf{12648}} &  {\bf{14112}} \\%
   {\BBlue{1921}} & 1707 & 1699 \\%
   {\BBlue{1731}} & 1636 &   \\%
   {\BRed{1678}} & {\bf{1585}} & {\bf{ }} \\%
   {\BBlue{94}} &   &   \\%
   \multicolumn{3}{@{\hspace{1pt}}c}{ }\\%
   \end{tabular*}
   \vfill}}
  \end{minipage}}}

%%  Rubidium
\newcommand{\ElemRb}{{%
  \begin{minipage}{67mm}%
  \vspace{1mm}

  {\Huge{\hspace{1mm} {\textbf{Rb}} \hfill \hfil{\textbf{37}} \hspace{1mm}}} %

  \vspace{6mm}

  {\Huge{\hfill {\Name{rubidium}} \hfill}}

  \vspace{6mm}

  {\Large{
  \begin{tabular*}{67mm}%
   {@{\hspace{5pt}}{r}@{\extracolsep{\fill}}r@{\extracolsep{\fill}}r}%
   %\multicolumn{3}{@{\hspace{1pt}}c}{ }\\%
   %\multicolumn{3}{c}{{\Huge{\BBlue{rubidium}}} }\\%
   % \multicolumn{3}{@{\hspace{1pt}}c}{ }\\%
   {\BRed{15200}}  & {\bf{13396}} &  {\bf{14961}} \\%
   {\BBlue{2065}} & 1826 & 1816 \\%
   {\BBlue{1864}} & 1751 &   \\%
   {\BRed{1804}} & {\bf{1692}} & {\bf{ }} \\%
   {\BBlue{112}} &   &   \\%
   \multicolumn{3}{@{\hspace{1pt}}c}{ }\\%
   \end{tabular*}
   \vfill}}
  \end{minipage}}}

%%  Strontium
\newcommand{\ElemSr}{{%
  \begin{minipage}{67mm}%
  \vspace{1mm}

  {\Huge{\hspace{1mm} {\textbf{Sr}} \hfill \hfil{\textbf{38}} \hspace{1mm}}} %

  \vspace{6mm}

  {\Huge{\hfill {\Name{strontium}} \hfill}}

  \vspace{6mm}

  {\Large{
  \begin{tabular*}{67mm}%
   {@{\hspace{5pt}}{r}@{\extracolsep{\fill}}r@{\extracolsep{\fill}}r}%
   %\multicolumn{3}{@{\hspace{1pt}}c}{ }\\%
   %\multicolumn{3}{c}{{\Huge{\BBlue{strontium}}} }\\%
   % \multicolumn{3}{@{\hspace{1pt}}c}{ }\\%
   {\BRed{16105}}  & {\bf{14165}} &  {\bf{15835}} \\%
   {\BBlue{2216}} & 1946 & 1936 \\%
   {\BBlue{2007}} & 1871 &   \\%
   {\BRed{1940}} & {\bf{1806}} & {\bf{ }} \\%
   {\BBlue{134}} &   &   \\%
   \multicolumn{3}{@{\hspace{1pt}}c}{ }\\%
   \end{tabular*}
   \vfill}}
  \end{minipage}}}

%%  Yttrium
\newcommand{\ElemY}{{%
  \begin{minipage}{67mm}%
  \vspace{1mm}

  {\Huge{\hspace{1mm} {\textbf{Y}} \hfill \hfil{\textbf{39}} \hspace{1mm}}} %

  \vspace{6mm}

  {\Huge{\hfill {\Name{yttrium}} \hfill}}

  \vspace{6mm}

  {\Large{
  \begin{tabular*}{67mm}%
   {@{\hspace{5pt}}{r}@{\extracolsep{\fill}}r@{\extracolsep{\fill}}r}%
   %\multicolumn{3}{@{\hspace{1pt}}c}{ }\\%
   %\multicolumn{3}{c}{{\Huge{\BBlue{yttrium}}} }\\%
   % \multicolumn{3}{@{\hspace{1pt}}c}{ }\\%
   {\BRed{17038}}  & {\bf{14958}} &  {\bf{16739}} \\%
   {\BBlue{2373}} & 2074 & 2062 \\%
   {\BBlue{2156}} & 1998 &   \\%
   {\BRed{2080}} & {\bf{1924}} & {\bf{ }} \\%
   {\BBlue{156}} &   &   \\%
   \multicolumn{3}{@{\hspace{1pt}}c}{ }\\%
   \end{tabular*}
   \vfill}}
  \end{minipage}}}

%%  Zirconium
\newcommand{\ElemZr}{{%
  \begin{minipage}{67mm}%
  \vspace{1mm}

  {\Huge{\hspace{1mm} {\textbf{Zr}} \hfill \hfil{\textbf{40}} \hspace{1mm}}} %

  \vspace{6mm}

  {\Huge{\hfill {\Name{zirconium}} \hfill}}

  \vspace{6mm}

  {\Large{
  \begin{tabular*}{67mm}%
   {@{\hspace{5pt}}{r}@{\extracolsep{\fill}}r@{\extracolsep{\fill}}r}%
   %\multicolumn{3}{@{\hspace{1pt}}c}{ }\\%
   %\multicolumn{3}{c}{{\Huge{\BBlue{zirconium}}} }\\%
   % \multicolumn{3}{@{\hspace{1pt}}c}{ }\\%
   {\BRed{17998}}  & {\bf{15775}} &  {\bf{17668}} \\%
   {\BBlue{2532}} & 2202 & 2189 \\%
   {\BBlue{2307}} & 2126 &   \\%
   {\BRed{2223}} & {\bf{2044}} & {\bf{ }} \\%
   {\BBlue{179}} &   &   \\%
   \multicolumn{3}{@{\hspace{1pt}}c}{ }\\%
   \end{tabular*}
   \vfill}}
  \end{minipage}}}

%%  Niobium
\newcommand{\ElemNb}{{%
  \begin{minipage}{67mm}%
  \vspace{1mm}

  {\Huge{\hspace{1mm} {\textbf{Nb}} \hfill \hfil{\textbf{41}} \hspace{1mm}}} %

  \vspace{6mm}

  {\Huge{\hfill {\Name{niobium}} \hfill}}

  \vspace{6mm}

  {\Large{
  \begin{tabular*}{67mm}%
   {@{\hspace{5pt}}{r}@{\extracolsep{\fill}}r@{\extracolsep{\fill}}r}%
   %\multicolumn{3}{@{\hspace{1pt}}c}{ }\\%
   %\multicolumn{3}{c}{{\Huge{\BBlue{niobium}}} }\\%
   % \multicolumn{3}{@{\hspace{1pt}}c}{ }\\%
   {\BRed{18986}}  & {\bf{16615}} &  {\bf{18625}} \\%
   {\BBlue{2698}} & 2337 & 2322 \\%
   {\BBlue{2465}} & 2260 &   \\%
   {\BRed{2371}} & {\bf{2169}} & {\bf{ }} \\%
   {\BBlue{202}} &   &   \\%
   \multicolumn{3}{@{\hspace{1pt}}c}{ }\\%
   \end{tabular*}
   \vfill}}
  \end{minipage}}}

%%  Molybdenum
\newcommand{\ElemMo}{{%
  \begin{minipage}{67mm}%
  \vspace{1mm}

  {\Huge{\hspace{1mm} {\textbf{Mo}} \hfill \hfil{\textbf{42}} \hspace{1mm}}} %

  \vspace{6mm}

  {\Huge{\hfill {\Name{molybdenum}} \hfill}}

  \vspace{6mm}

  {\Large{
  \begin{tabular*}{67mm}%
   {@{\hspace{5pt}}{r}@{\extracolsep{\fill}}r@{\extracolsep{\fill}}r}%
   %\multicolumn{3}{@{\hspace{1pt}}c}{ }\\%
   %\multicolumn{3}{c}{{\Huge{\BBlue{molybdenum}}} }\\%
   % \multicolumn{3}{@{\hspace{1pt}}c}{ }\\%
   {\BRed{20000}}  & {\bf{17480}} &  {\bf{19606}} \\%
   {\BBlue{2866}} & 2472 & 2454 \\%
   {\BBlue{2625}} & 2394 &   \\%
   {\BRed{2520}} & {\bf{2292}} & {\bf{ }} \\%
   {\BBlue{228}} &   &   \\%
   \multicolumn{3}{@{\hspace{1pt}}c}{ }\\%
   \end{tabular*}
   \vfill}}
  \end{minipage}}}

%%  Technetium
\newcommand{\ElemTc}{{%
  \begin{minipage}{67mm}%
  \vspace{1mm}

  {\Huge{\hspace{1mm} {\textbf{Tc}} \hfill \hfil{\textbf{43}} \hspace{1mm}}} %

  \vspace{6mm}

  {\Huge{\hfill {\Name{technetium}} \hfill}}

  \vspace{6mm}

  {\Large{
  \begin{tabular*}{67mm}%
   {@{\hspace{5pt}}{r}@{\extracolsep{\fill}}r@{\extracolsep{\fill}}r}%
   %\multicolumn{3}{@{\hspace{1pt}}c}{ }\\%
   %\multicolumn{3}{c}{{\Huge{\BBlue{technetium}}} }\\%
   % \multicolumn{3}{@{\hspace{1pt}}c}{ }\\%
   {\BRed{21044}}  & {\bf{18367}} &  {\bf{20626}} \\%
   {\BBlue{3043}} & 2625 & 2595 \\%
   {\BBlue{2793}} & 2535 &   \\%
   {\BRed{2677}} & {\bf{2423}} & {\bf{ }} \\%
   {\BBlue{254}} &   &   \\%
   \multicolumn{3}{@{\hspace{1pt}}c}{ }\\%
   \end{tabular*}
   \vfill}}
  \end{minipage}}}

%%  Ruthenium
\newcommand{\ElemRu}{{%
  \begin{minipage}{67mm}%
  \vspace{1mm}

  {\Huge{\hspace{1mm} {\textbf{Ru}} \hfill \hfil{\textbf{44}} \hspace{1mm}}} %

  \vspace{6mm}

  {\Huge{\hfill {\Name{ruthenium}} \hfill}}

  \vspace{6mm}

  {\Large{
  \begin{tabular*}{67mm}%
   {@{\hspace{5pt}}{r}@{\extracolsep{\fill}}r@{\extracolsep{\fill}}r}%
   %\multicolumn{3}{@{\hspace{1pt}}c}{ }\\%
   %\multicolumn{3}{c}{{\Huge{\BBlue{ruthenium}}} }\\%
   % \multicolumn{3}{@{\hspace{1pt}}c}{ }\\%
   {\BRed{22117}}  & {\bf{19279}} &  {\bf{21656}} \\%
   {\BBlue{3224}} & 2763 & 2741 \\%
   {\BBlue{2967}} & 2683 &   \\%
   {\BRed{2838}} & {\bf{2558}} & {\bf{ }} \\%
   {\BBlue{280}} &   &   \\%
   \multicolumn{3}{@{\hspace{1pt}}c}{ }\\%
   \end{tabular*}
   \vfill}}
  \end{minipage}}}

%%  Rhodium
\newcommand{\ElemRh}{{%
  \begin{minipage}{67mm}%
  \vspace{1mm}

  {\Huge{\hspace{1mm} {\textbf{Rh}} \hfill \hfil{\textbf{45}} \hspace{1mm}}} %

  \vspace{6mm}

  {\Huge{\hfill {\Name{rhodium}} \hfill}}

  \vspace{6mm}

  {\Large{
  \begin{tabular*}{67mm}%
   {@{\hspace{5pt}}{r}@{\extracolsep{\fill}}r@{\extracolsep{\fill}}r}%
   %\multicolumn{3}{@{\hspace{1pt}}c}{ }\\%
   %\multicolumn{3}{c}{{\Huge{\BBlue{rhodium}}} }\\%
   % \multicolumn{3}{@{\hspace{1pt}}c}{ }\\%
   {\BRed{23220}}  & {\bf{20216}} &  {\bf{22724}} \\%
   {\BBlue{3412}} & 2916 & 2891 \\%
   {\BBlue{3146}} & 2834 & 3144 \\%
   {\BRed{3004}} & {\bf{2697}} & {\bf{3002}} \\%
   {\BBlue{307}} &   &   \\%
   \multicolumn{3}{@{\hspace{1pt}}c}{ }\\%
   \end{tabular*}
   \vfill}}
  \end{minipage}}}

%%  Palladium
\newcommand{\ElemPd}{{%
  \begin{minipage}{67mm}%
  \vspace{1mm}

  {\Huge{\hspace{1mm} {\textbf{Pd}} \hfill \hfil{\textbf{46}} \hspace{1mm}}} %

  \vspace{6mm}

  {\Huge{\hfill {\Name{palladium}} \hfill}}

  \vspace{6mm}

  {\Large{
  \begin{tabular*}{67mm}%
   {@{\hspace{5pt}}{r}@{\extracolsep{\fill}}r@{\extracolsep{\fill}}r}%
   %\multicolumn{3}{@{\hspace{1pt}}c}{ }\\%
   %\multicolumn{3}{c}{{\Huge{\BBlue{palladium}}} }\\%
   % \multicolumn{3}{@{\hspace{1pt}}c}{ }\\%
   {\BRed{24350}}  & {\bf{21177}} &  {\bf{23818}} \\%
   {\BBlue{3604}} & 3072 & 3044 \\%
   {\BBlue{3330}} & 2990 & 3328 \\%
   {\BRed{3173}} & {\bf{2838}} & {\bf{3171}} \\%
   {\BBlue{335}} &   &   \\%
   \multicolumn{3}{@{\hspace{1pt}}c}{ }\\%
   \end{tabular*}
   \vfill}}
  \end{minipage}}}

%%  Silver
\newcommand{\ElemAg}{{%
  \begin{minipage}{67mm}%
  \vspace{1mm}

  {\Huge{\hspace{1mm} {\textbf{Ag}} \hfill \hfil{\textbf{47}} \hspace{1mm}}} %

  \vspace{6mm}

  {\Huge{\hfill {\Name{silver}} \hfill}}

  \vspace{6mm}

  {\Large{
  \begin{tabular*}{67mm}%
   {@{\hspace{5pt}}{r}@{\extracolsep{\fill}}r@{\extracolsep{\fill}}r}%
   %\multicolumn{3}{@{\hspace{1pt}}c}{ }\\%
   %\multicolumn{3}{c}{{\Huge{\BBlue{silver}}} }\\%
   % \multicolumn{3}{@{\hspace{1pt}}c}{ }\\%
   {\BRed{25514}}  & {\bf{22163}} &  {\bf{24941}} \\%
   {\BBlue{3806}} & 3233 & 3202 \\%
   {\BBlue{3524}} & 3150 & 3520 \\%
   {\BRed{3351}} & {\bf{2983}} & {\bf{3347}} \\%
   {\BBlue{368}} &   &   \\%
   \multicolumn{3}{@{\hspace{1pt}}c}{ }\\%
   \end{tabular*}
   \vfill}}
  \end{minipage}}}

%%  Cadmium
\newcommand{\ElemCd}{{%
  \begin{minipage}{67mm}%
  \vspace{1mm}

  {\Huge{\hspace{1mm} {\textbf{Cd}} \hfill \hfil{\textbf{48}} \hspace{1mm}}} %

  \vspace{6mm}

  {\Huge{\hfill {\Name{cadmium}} \hfill}}

  \vspace{6mm}

  {\Large{
  \begin{tabular*}{67mm}%
   {@{\hspace{5pt}}{r}@{\extracolsep{\fill}}r@{\extracolsep{\fill}}r}%
   %\multicolumn{3}{@{\hspace{1pt}}c}{ }\\%
   %\multicolumn{3}{c}{{\Huge{\BBlue{cadmium}}} }\\%
   % \multicolumn{3}{@{\hspace{1pt}}c}{ }\\%
   {\BRed{26711}}  & {\bf{23173}} &  {\bf{26093}} \\%
   {\BBlue{4018}} & 3400 & 3365 \\%
   {\BBlue{3727}} & 3315 & 3715 \\%
   {\BRed{3538}} & {\bf{3133}} & {\bf{3526}} \\%
   {\BBlue{405}} &   &   \\%
   \multicolumn{3}{@{\hspace{1pt}}c}{ }\\%
   \end{tabular*}
   \vfill}}
  \end{minipage}}}

%%  Indium
\newcommand{\ElemIn}{{%
  \begin{minipage}{67mm}%
  \vspace{1mm}

  {\Huge{\hspace{1mm} {\textbf{In}} \hfill \hfil{\textbf{49}} \hspace{1mm}}} %

  \vspace{6mm}

  {\Huge{\hfill {\Name{indium}} \hfill}}

  \vspace{6mm}

  {\Large{
  \begin{tabular*}{67mm}%
   {@{\hspace{5pt}}{r}@{\extracolsep{\fill}}r@{\extracolsep{\fill}}r}%
   %\multicolumn{3}{@{\hspace{1pt}}c}{ }\\%
   %\multicolumn{3}{c}{{\Huge{\BBlue{indium}}} }\\%
   % \multicolumn{3}{@{\hspace{1pt}}c}{ }\\%
   {\BRed{27940}}  & {\bf{24210}} &  {\bf{27275}} \\%
   {\BBlue{4238}} & 3573 & 3535 \\%
   {\BBlue{3938}} & 3487 & 3920 \\%
   {\BRed{3730}} & {\bf{3286}} & {\bf{3712}} \\%
   {\BBlue{444}} &   &   \\%
   \multicolumn{3}{@{\hspace{1pt}}c}{ }\\%
   \end{tabular*}
   \vfill}}
  \end{minipage}}}

%%  Tin
\newcommand{\ElemSn}{{%
  \begin{minipage}{67mm}%
  \vspace{1mm}

  {\Huge{\hspace{1mm} {\textbf{Sn}} \hfill \hfil{\textbf{50}} \hspace{1mm}}} %

  \vspace{6mm}

  {\Huge{\hfill {\Name{tin}} \hfill}}

  \vspace{6mm}

  {\Large{
  \begin{tabular*}{67mm}%
   {@{\hspace{5pt}}{r}@{\extracolsep{\fill}}r@{\extracolsep{\fill}}r}%
   %\multicolumn{3}{@{\hspace{1pt}}c}{ }\\%
   %\multicolumn{3}{c}{{\Huge{\BBlue{tin}}} }\\%
   % \multicolumn{3}{@{\hspace{1pt}}c}{ }\\%
   {\BRed{29200}}  & {\bf{25271}} &  {\bf{28485}} \\%
   {\BBlue{4465}} & 3750 & 3709 \\%
   {\BBlue{4156}} & 3663 & 4131 \\%
   {\BRed{3929}} & {\bf{3444}} & {\bf{3904}} \\%
   {\BBlue{485}} &   &   \\%
   \multicolumn{3}{@{\hspace{1pt}}c}{ }\\%
   \end{tabular*}
   \vfill}}
  \end{minipage}}}

%%  Antimony
\newcommand{\ElemSb}{{%
  \begin{minipage}{67mm}%
  \vspace{1mm}

  {\Huge{\hspace{1mm} {\textbf{Sb}} \hfill \hfil{\textbf{51}} \hspace{1mm}}} %

  \vspace{6mm}

  {\Huge{\hfill {\Name{antimony}} \hfill}}

  \vspace{6mm}

  {\Large{
  \begin{tabular*}{67mm}%
   {@{\hspace{5pt}}{r}@{\extracolsep{\fill}}r@{\extracolsep{\fill}}r}%
   %\multicolumn{3}{@{\hspace{1pt}}c}{ }\\%
   %\multicolumn{3}{c}{{\Huge{\BBlue{antimony}}} }\\%
   % \multicolumn{3}{@{\hspace{1pt}}c}{ }\\%
   {\BRed{30491}}  & {\bf{26359}} &  {\bf{29725}} \\%
   {\BBlue{4698}} & 3932 & 3885 \\%
   {\BBlue{4380}} & 3843 & 4347 \\%
   {\BRed{4132}} & {\bf{3604}} & {\bf{4099}} \\%
   {\BBlue{528}} & 528 & 538 \\%
   \multicolumn{3}{@{\hspace{1pt}}c}{ }\\%
   \end{tabular*}
   \vfill}}
  \end{minipage}}}

%%  Tellurium
\newcommand{\ElemTe}{{%
  \begin{minipage}{67mm}%
  \vspace{1mm}

  {\Huge{\hspace{1mm} {\textbf{Te}} \hfill \hfil{\textbf{52}} \hspace{1mm}}} %

  \vspace{6mm}

  {\Huge{\hfill {\Name{tellurium}} \hfill}}

  \vspace{6mm}

  {\Large{
  \begin{tabular*}{67mm}%
   {@{\hspace{5pt}}{r}@{\extracolsep{\fill}}r@{\extracolsep{\fill}}r}%
   %\multicolumn{3}{@{\hspace{1pt}}c}{ }\\%
   %\multicolumn{3}{c}{{\Huge{\BBlue{tellurium}}} }\\%
   % \multicolumn{3}{@{\hspace{1pt}}c}{ }\\%
   {\BRed{31814}}  & {\bf{27473}} &  {\bf{30993}} \\%
   {\BBlue{4939}} & 4118 & 4068 \\%
   {\BBlue{4612}} & 4029 & 4570 \\%
   {\BRed{4341}} & {\bf{3768}} & {\bf{4299}} \\%
   {\BBlue{573}} & 573 & 583 \\%
   \multicolumn{3}{@{\hspace{1pt}}c}{ }\\%
   \end{tabular*}
   \vfill}}
  \end{minipage}}}

%%  Iodine
\newcommand{\ElemI}{{%
  \begin{minipage}{67mm}%
  \vspace{1mm}

  {\Huge{\hspace{1mm} {\textbf{I}} \hfill \hfil{\textbf{53}} \hspace{1mm}}} %

  \vspace{6mm}

  {\Huge{\hfill {\Name{iodine}} \hfill}}

  \vspace{6mm}

  {\Large{
  \begin{tabular*}{67mm}%
   {@{\hspace{5pt}}{r}@{\extracolsep{\fill}}r@{\extracolsep{\fill}}r}%
   %\multicolumn{3}{@{\hspace{1pt}}c}{ }\\%
   %\multicolumn{3}{c}{{\Huge{\BBlue{iodine}}} }\\%
   % \multicolumn{3}{@{\hspace{1pt}}c}{ }\\%
   {\BRed{33169}}  & {\bf{28612}} &  {\bf{32294}} \\%
   {\BBlue{5188}} & 4313 & 4257 \\%
   {\BBlue{4852}} & 4221 & 4801 \\%
   {\BRed{4557}} & {\bf{3938}} & {\bf{4506}} \\%
   {\BBlue{619}} & 619 & 631 \\%
   \multicolumn{3}{@{\hspace{1pt}}c}{ }\\%
   \end{tabular*}
   \vfill}}
  \end{minipage}}}

%%  Xenon
\newcommand{\ElemXe}{{%
  \begin{minipage}{67mm}%
  \vspace{1mm}

  {\Huge{\hspace{1mm} {\textbf{Xe}} \hfill \hfil{\textbf{54}} \hspace{1mm}}} %

  \vspace{6mm}

  {\Huge{\hfill {\Name{xenon}} \hfill}}

  \vspace{6mm}

  {\Large{
  \begin{tabular*}{67mm}%
   {@{\hspace{5pt}}{r}@{\extracolsep{\fill}}r@{\extracolsep{\fill}}r}%
   %\multicolumn{3}{@{\hspace{1pt}}c}{ }\\%
   %\multicolumn{3}{c}{{\Huge{\BBlue{xenon}}} }\\%
   % \multicolumn{3}{@{\hspace{1pt}}c}{ }\\%
   {\BRed{34561}}  & {\bf{29775}} &  {\bf{33620}} \\%
   {\BBlue{5453}} & 4512 & 4451 \\%
   {\BBlue{5107}} & 4418 & 5038 \\%
   {\BRed{4786}} & {\bf{4110}} & {\bf{4717}} \\%
   {\BBlue{676}} & 676 & 689 \\%
   \multicolumn{3}{@{\hspace{1pt}}c}{ }\\%
   \end{tabular*}
   \vfill}}
  \end{minipage}}}

%%  Cesium
\newcommand{\ElemCs}{{%
  \begin{minipage}{67mm}%
  \vspace{1mm}

  {\Huge{\hspace{1mm} {\textbf{Cs}} \hfill \hfil{\textbf{55}} \hspace{1mm}}} %

  \vspace{6mm}

  {\Huge{\hfill {\Name{cesium}} \hfill}}

  \vspace{6mm}

  {\Large{
  \begin{tabular*}{67mm}%
   {@{\hspace{5pt}}{r}@{\extracolsep{\fill}}r@{\extracolsep{\fill}}r}%
   %\multicolumn{3}{@{\hspace{1pt}}c}{ }\\%
   %\multicolumn{3}{c}{{\Huge{\BBlue{cesium}}} }\\%
   % \multicolumn{3}{@{\hspace{1pt}}c}{ }\\%
   {\BRed{35985}}  & {\bf{30973}} &  {\bf{34982}} \\%
   {\BBlue{5714}} & 4711 & 4643 \\%
   {\BBlue{5359}} & 4619 & 5279 \\%
   {\BRed{5012}} & {\bf{4285}} & {\bf{4932}} \\%
   {\BBlue{727}} & 727 & 741 \\%
   \multicolumn{3}{@{\hspace{1pt}}c}{ }\\%
   \end{tabular*}
   \vfill}}
  \end{minipage}}}

%%  Barium
\newcommand{\ElemBa}{{%
  \begin{minipage}{67mm}%
  \vspace{1mm}

  {\Huge{\hspace{1mm} {\textbf{Ba}} \hfill \hfil{\textbf{56}} \hspace{1mm}}} %

  \vspace{6mm}

  {\Huge{\hfill {\Name{barium}} \hfill}}

  \vspace{6mm}

  {\Large{
  \begin{tabular*}{67mm}%
   {@{\hspace{5pt}}{r}@{\extracolsep{\fill}}r@{\extracolsep{\fill}}r}%
   %\multicolumn{3}{@{\hspace{1pt}}c}{ }\\%
   %\multicolumn{3}{c}{{\Huge{\BBlue{barium}}} }\\%
   % \multicolumn{3}{@{\hspace{1pt}}c}{ }\\%
   {\BRed{37441}}  & {\bf{32194}} &  {\bf{36378}} \\%
   {\BBlue{5989}} & 4926 & 4852 \\%
   {\BBlue{5624}} & 4828 & 5531 \\%
   {\BRed{5247}} & {\bf{4467}} & {\bf{5154}} \\%
   {\BBlue{781}} & 781 & 796 \\%
   \multicolumn{3}{@{\hspace{1pt}}c}{ }\\%
   \end{tabular*}
   \vfill}}
  \end{minipage}}}

%%  Lanthanum
\newcommand{\ElemLa}{{%
  \begin{minipage}{67mm}%
  \vspace{1mm}

  {\Huge{\hspace{1mm} {\textbf{La}} \hfill \hfil{\textbf{57}} \hspace{1mm}}} %

  \vspace{6mm}

  {\Huge{\hfill {\Name{lanthanum}} \hfill}}

  \vspace{6mm}

  {\Large{
  \begin{tabular*}{67mm}%
   {@{\hspace{5pt}}{r}@{\extracolsep{\fill}}r@{\extracolsep{\fill}}r}%
   %\multicolumn{3}{@{\hspace{1pt}}c}{ }\\%
   %\multicolumn{3}{c}{{\Huge{\BBlue{lanthanum}}} }\\%
   % \multicolumn{3}{@{\hspace{1pt}}c}{ }\\%
   {\BRed{38925}}  & {\bf{33442}} &  {\bf{37797}} \\%
   {\BBlue{6266}} & 5138 & 5057 \\%
   {\BBlue{5891}} & 5038 & 5786 \\%
   {\BRed{5483}} & {\bf{4647}} & {\bf{5378}} \\%
   {\BBlue{836}} & 836 & 853 \\%
   \multicolumn{3}{@{\hspace{1pt}}c}{ }\\%
   \end{tabular*}
   \vfill}}
  \end{minipage}}}

%%  Cerium
\newcommand{\ElemCe}{{%
  \begin{minipage}{67mm}%
  \vspace{1mm}

  {\Huge{\hspace{1mm} {\textbf{Ce}} \hfill \hfil{\textbf{58}} \hspace{1mm}}} %

  \vspace{6mm}

  {\Huge{\hfill {\Name{cerium}} \hfill}}

  \vspace{6mm}

  {\Large{
  \begin{tabular*}{67mm}%
   {@{\hspace{5pt}}{r}@{\extracolsep{\fill}}r@{\extracolsep{\fill}}r}%
   %\multicolumn{3}{@{\hspace{1pt}}c}{ }\\%
   %\multicolumn{3}{c}{{\Huge{\BBlue{cerium}}} }\\%
   % \multicolumn{3}{@{\hspace{1pt}}c}{ }\\%
   {\BRed{40443}}  & {\bf{34720}} &  {\bf{39256}} \\%
   {\BBlue{6548}} & 5361 & 5274 \\%
   {\BBlue{6164}} & 5262 & 6055 \\%
   {\BRed{5723}} & {\bf{4839}} & {\bf{5614}} \\%
   {\BBlue{884}} & 884 & 902 \\%
   \multicolumn{3}{@{\hspace{1pt}}c}{ }\\%
   \end{tabular*}
   \vfill}}
  \end{minipage}}}

%%  Praseodymium
\newcommand{\ElemPr}{{%
  \begin{minipage}{67mm}%
  \vspace{1mm}

  {\Huge{\hspace{1mm} {\textbf{Pr}} \hfill \hfil{\textbf{59}} \hspace{1mm}}} %

  \vspace{6mm}

  {\Huge{\hfill {\Name{praseodymium}} \hfill}}

  \vspace{6mm}

  {\Large{
  \begin{tabular*}{67mm}%
   {@{\hspace{5pt}}{r}@{\extracolsep{\fill}}r@{\extracolsep{\fill}}r}%
   %\multicolumn{3}{@{\hspace{1pt}}c}{ }\\%
   %\multicolumn{3}{c}{{\Huge{\BBlue{praseodymium}}} }\\%
   % \multicolumn{3}{@{\hspace{1pt}}c}{ }\\%
   {\BRed{41991}}  & {\bf{36027}} &  {\bf{40749}} \\%
   {\BBlue{6835}} & 5593 & 5498 \\%
   {\BBlue{6440}} & 5492 & 6325 \\%
   {\BRed{5964}} & {\bf{5035}} & {\bf{5849}} \\%
   {\BBlue{929}} & 927 & 946 \\%
   \multicolumn{3}{@{\hspace{1pt}}c}{ }\\%
   \end{tabular*}
   \vfill}}
  \end{minipage}}}

%%  Neodymium
\newcommand{\ElemNd}{{%
  \begin{minipage}{67mm}%
  \vspace{1mm}

  {\Huge{\hspace{1mm} {\textbf{Nd}} \hfill \hfil{\textbf{60}} \hspace{1mm}}} %

  \vspace{6mm}

  {\Huge{\hfill {\Name{neodymium}} \hfill}}

  \vspace{6mm}

  {\Large{
  \begin{tabular*}{67mm}%
   {@{\hspace{5pt}}{r}@{\extracolsep{\fill}}r@{\extracolsep{\fill}}r}%
   %\multicolumn{3}{@{\hspace{1pt}}c}{ }\\%
   %\multicolumn{3}{c}{{\Huge{\BBlue{neodymium}}} }\\%
   % \multicolumn{3}{@{\hspace{1pt}}c}{ }\\%
   {\BRed{43569}}  & {\bf{37361}} &  {\bf{42272}} \\%
   {\BBlue{7126}} & 5829 & 5723 \\%
   {\BBlue{6722}} & 5719 & 6602 \\%
   {\BRed{6208}} & {\bf{5228}} & {\bf{6088}} \\%
   {\BBlue{980}} & 979 & 1002 \\%
   \multicolumn{3}{@{\hspace{1pt}}c}{ }\\%
   \end{tabular*}
   \vfill}}
  \end{minipage}}}

%%  Promethium
\newcommand{\ElemPm}{{%
  \begin{minipage}{67mm}%
  \vspace{1mm}

  {\Huge{\hspace{1mm} {\textbf{Pm}} \hfill \hfil{\textbf{61}} \hspace{1mm}}} %

  \vspace{6mm}

  {\Huge{\hfill {\Name{promethium}} \hfill}}

  \vspace{6mm}

  {\Large{
  \begin{tabular*}{67mm}%
   {@{\hspace{5pt}}{r}@{\extracolsep{\fill}}r@{\extracolsep{\fill}}r}%
   %\multicolumn{3}{@{\hspace{1pt}}c}{ }\\%
   %\multicolumn{3}{c}{{\Huge{\BBlue{promethium}}} }\\%
   % \multicolumn{3}{@{\hspace{1pt}}c}{ }\\%
   {\BRed{45184}}  & {\bf{38725}} &  {\bf{43827}} \\%
   {\BBlue{7428}} & 6071 & 5957 \\%
   {\BBlue{7013}} & 5961 & 6893 \\%
   {\BRed{6459}} & {\bf{5432}} & {\bf{6339}} \\%
   {\BBlue{1027}} & 1023 & 1048 \\%
   \multicolumn{3}{@{\hspace{1pt}}c}{ }\\%
   \end{tabular*}
   \vfill}}
  \end{minipage}}}

%%  Samarium
\newcommand{\ElemSm}{{%
  \begin{minipage}{67mm}%
  \vspace{1mm}

  {\Huge{\hspace{1mm} {\textbf{Sm}} \hfill \hfil{\textbf{62}} \hspace{1mm}}} %

  \vspace{6mm}

  {\Huge{\hfill {\Name{samarium}} \hfill}}

  \vspace{6mm}

  {\Large{
  \begin{tabular*}{67mm}%
   {@{\hspace{5pt}}{r}@{\extracolsep{\fill}}r@{\extracolsep{\fill}}r}%
   %\multicolumn{3}{@{\hspace{1pt}}c}{ }\\%
   %\multicolumn{3}{c}{{\Huge{\BBlue{samarium}}} }\\%
   % \multicolumn{3}{@{\hspace{1pt}}c}{ }\\%
   {\BRed{46834}}  & {\bf{40118}} &  {\bf{45414}} \\%
   {\BBlue{7737}} & 6317 & 6196 \\%
   {\BBlue{7312}} & 6201 & 7183 \\%
   {\BRed{6716}} & {\bf{5633}} & {\bf{6587}} \\%
   {\BBlue{1083}} & 1078 & 1106 \\%
   \multicolumn{3}{@{\hspace{1pt}}c}{ }\\%
   \end{tabular*}
   \vfill}}
  \end{minipage}}}

%%  Europium
\newcommand{\ElemEu}{{%
  \begin{minipage}{67mm}%
  \vspace{1mm}

  {\Huge{\hspace{1mm} {\textbf{Eu}} \hfill \hfil{\textbf{63}} \hspace{1mm}}} %

  \vspace{6mm}

  {\Huge{\hfill {\Name{europium}} \hfill}}

  \vspace{6mm}

  {\Large{
  \begin{tabular*}{67mm}%
   {@{\hspace{5pt}}{r}@{\extracolsep{\fill}}r@{\extracolsep{\fill}}r}%
   %\multicolumn{3}{@{\hspace{1pt}}c}{ }\\%
   %\multicolumn{3}{c}{{\Huge{\BBlue{europium}}} }\\%
   % \multicolumn{3}{@{\hspace{1pt}}c}{ }\\%
   {\BRed{48519}}  & {\bf{41542}} &  {\bf{47038}} \\%
   {\BBlue{8052}} & 6571 & 6438 \\%
   {\BBlue{7617}} & 6458 & 7484 \\%
   {\BRed{6977}} & {\bf{5850}} & {\bf{6844}} \\%
   {\BBlue{1128}} & 1122 & 1153 \\%
   \multicolumn{3}{@{\hspace{1pt}}c}{ }\\%
   \end{tabular*}
   \vfill}}
  \end{minipage}}}

%%  Gadolinium
\newcommand{\ElemGd}{{%
  \begin{minipage}{67mm}%
  \vspace{1mm}

  {\Huge{\hspace{1mm} {\textbf{Gd}} \hfill \hfil{\textbf{64}} \hspace{1mm}}} %

  \vspace{6mm}

  {\Huge{\hfill {\Name{gadolinium}} \hfill}}

  \vspace{6mm}

  {\Large{
  \begin{tabular*}{67mm}%
   {@{\hspace{5pt}}{r}@{\extracolsep{\fill}}r@{\extracolsep{\fill}}r}%
   %\multicolumn{3}{@{\hspace{1pt}}c}{ }\\%
   %\multicolumn{3}{c}{{\Huge{\BBlue{gadolinium}}} }\\%
   % \multicolumn{3}{@{\hspace{1pt}}c}{ }\\%
   {\BRed{50239}}  & {\bf{42996}} &  {\bf{48695}} \\%
   {\BBlue{8376}} & 6832 & 6688 \\%
   {\BBlue{7930}} & 6708 & 7787 \\%
   {\BRed{7243}} & {\bf{6053}} & {\bf{7100}} \\%
   {\BBlue{1190}} & 1181 & 1213 \\%
   \multicolumn{3}{@{\hspace{1pt}}c}{ }\\%
   \end{tabular*}
   \vfill}}
  \end{minipage}}}

%%  Terbium
\newcommand{\ElemTb}{{%
  \begin{minipage}{67mm}%
  \vspace{1mm}

  {\Huge{\hspace{1mm} {\textbf{Tb}} \hfill \hfil{\textbf{65}} \hspace{1mm}}} %

  \vspace{6mm}

  {\Huge{\hfill {\Name{terbium}} \hfill}}

  \vspace{6mm}

  {\Large{
  \begin{tabular*}{67mm}%
   {@{\hspace{5pt}}{r}@{\extracolsep{\fill}}r@{\extracolsep{\fill}}r}%
   %\multicolumn{3}{@{\hspace{1pt}}c}{ }\\%
   %\multicolumn{3}{c}{{\Huge{\BBlue{terbium}}} }\\%
   % \multicolumn{3}{@{\hspace{1pt}}c}{ }\\%
   {\BRed{51996}}  & {\bf{44482}} &  {\bf{50385}} \\%
   {\BBlue{8708}} & 7097 & 6940 \\%
   {\BBlue{8252}} & 6975 & 8102 \\%
   {\BRed{7514}} & {\bf{6273}} & {\bf{7364}} \\%
   {\BBlue{1241}} & 1233 & 1269 \\%
   \multicolumn{3}{@{\hspace{1pt}}c}{ }\\%
   \end{tabular*}
   \vfill}}
  \end{minipage}}}

%%  Dysprosium
\newcommand{\ElemDy}{{%
  \begin{minipage}{67mm}%
  \vspace{1mm}

  {\Huge{\hspace{1mm} {\textbf{Dy}} \hfill \hfil{\textbf{66}} \hspace{1mm}}} %

  \vspace{6mm}

  {\Huge{\hfill {\Name{dysprosium}} \hfill}}

  \vspace{6mm}

  {\Large{
  \begin{tabular*}{67mm}%
   {@{\hspace{5pt}}{r}@{\extracolsep{\fill}}r@{\extracolsep{\fill}}r}%
   %\multicolumn{3}{@{\hspace{1pt}}c}{ }\\%
   %\multicolumn{3}{c}{{\Huge{\BBlue{dysprosium}}} }\\%
   % \multicolumn{3}{@{\hspace{1pt}}c}{ }\\%
   {\BRed{53789}}  & {\bf{45999}} &  {\bf{52113}} \\%
   {\BBlue{9046}} & 7370 & 7204 \\%
   {\BBlue{8581}} & 7248 & 8427 \\%
   {\BRed{7790}} & {\bf{6498}} & {\bf{7636}} \\%
   {\BBlue{1292}} & 1284 & 1325 \\%
   \multicolumn{3}{@{\hspace{1pt}}c}{ }\\%
   \end{tabular*}
   \vfill}}
  \end{minipage}}}

%%  Holmium
\newcommand{\ElemHo}{{%
  \begin{minipage}{67mm}%
  \vspace{1mm}

  {\Huge{\hspace{1mm} {\textbf{Ho}} \hfill \hfil{\textbf{67}} \hspace{1mm}}} %

  \vspace{6mm}

  {\Huge{\hfill {\Name{holmium}} \hfill}}

  \vspace{6mm}

  {\Large{
  \begin{tabular*}{67mm}%
   {@{\hspace{5pt}}{r}@{\extracolsep{\fill}}r@{\extracolsep{\fill}}r}%
   %\multicolumn{3}{@{\hspace{1pt}}c}{ }\\%
   %\multicolumn{3}{c}{{\Huge{\BBlue{holmium}}} }\\%
   % \multicolumn{3}{@{\hspace{1pt}}c}{ }\\%
   {\BRed{55618}}  & {\bf{47547}} &  {\bf{53877}} \\%
   {\BBlue{9394}} & 7653 & 7471 \\%
   {\BBlue{8918}} & 7526 & 8758 \\%
   {\BRed{8071}} & {\bf{6720}} & {\bf{7911}} \\%
   {\BBlue{1351}} & 1342 & 1383 \\%
   \multicolumn{3}{@{\hspace{1pt}}c}{ }\\%
   \end{tabular*}
   \vfill}}
  \end{minipage}}}

%%  Erbium
\newcommand{\ElemEr}{{%
  \begin{minipage}{67mm}%
  \vspace{1mm}

  {\Huge{\hspace{1mm} {\textbf{Er}} \hfill \hfil{\textbf{68}} \hspace{1mm}}} %

  \vspace{6mm}

  {\Huge{\hfill {\Name{erbium}} \hfill}}

  \vspace{6mm}

  {\Large{
  \begin{tabular*}{67mm}%
   {@{\hspace{5pt}}{r}@{\extracolsep{\fill}}r@{\extracolsep{\fill}}r}%
   %\multicolumn{3}{@{\hspace{1pt}}c}{ }\\%
   %\multicolumn{3}{c}{{\Huge{\BBlue{erbium}}} }\\%
   % \multicolumn{3}{@{\hspace{1pt}}c}{ }\\%
   {\BRed{57486}}  & {\bf{49128}} &  {\bf{55674}} \\%
   {\BBlue{9751}} & 7939 & 7745 \\%
   {\BBlue{9264}} & 7811 & 9096 \\%
   {\BRed{8358}} & {\bf{6949}} & {\bf{8190}} \\%
   {\BBlue{1409}} & 1404 & 1448 \\%
   \multicolumn{3}{@{\hspace{1pt}}c}{ }\\%
   \end{tabular*}
   \vfill}}
  \end{minipage}}}

%%  Thulium
\newcommand{\ElemTm}{{%
  \begin{minipage}{67mm}%
  \vspace{1mm}

  {\Huge{\hspace{1mm} {\textbf{Tm}} \hfill \hfil{\textbf{69}} \hspace{1mm}}} %

  \vspace{6mm}

  {\Huge{\hfill {\Name{thulium}} \hfill}}

  \vspace{6mm}

  {\Large{
  \begin{tabular*}{67mm}%
   {@{\hspace{5pt}}{r}@{\extracolsep{\fill}}r@{\extracolsep{\fill}}r}%
   %\multicolumn{3}{@{\hspace{1pt}}c}{ }\\%
   %\multicolumn{3}{c}{{\Huge{\BBlue{thulium}}} }\\%
   % \multicolumn{3}{@{\hspace{1pt}}c}{ }\\%
   {\BRed{59390}}  & {\bf{50742}} &  {\bf{57505}} \\%
   {\BBlue{10116}} & 8231 & 8026 \\%
   {\BBlue{9617}} & 8102 & 9442 \\%
   {\BRed{8648}} & {\bf{7180}} & {\bf{8473}} \\%
   {\BBlue{1468}} & 1463 & 1510 \\%
   \multicolumn{3}{@{\hspace{1pt}}c}{ }\\%
   \end{tabular*}
   \vfill}}
  \end{minipage}}}

%%  Ytterbium
\newcommand{\ElemYb}{{%
  \begin{minipage}{67mm}%
  \vspace{1mm}

  {\Huge{\hspace{1mm} {\textbf{Yb}} \hfill \hfil{\textbf{70}} \hspace{1mm}}} %

  \vspace{6mm}

  {\Huge{\hfill {\Name{ytterbium}} \hfill}}

  \vspace{6mm}

  {\Large{
  \begin{tabular*}{67mm}%
   {@{\hspace{5pt}}{r}@{\extracolsep{\fill}}r@{\extracolsep{\fill}}r}%
   %\multicolumn{3}{@{\hspace{1pt}}c}{ }\\%
   %\multicolumn{3}{c}{{\Huge{\BBlue{ytterbium}}} }\\%
   % \multicolumn{3}{@{\hspace{1pt}}c}{ }\\%
   {\BRed{61332}}  & {\bf{52388}} &  {\bf{59382}} \\%
   {\BBlue{10486}} & 8536 & 8313 \\%
   {\BBlue{9978}} & 8402 & 9787 \\%
   {\BRed{8944}} & {\bf{7416}} & {\bf{8753}} \\%
   {\BBlue{1528}} & 1526 & 1574 \\%
   \multicolumn{3}{@{\hspace{1pt}}c}{ }\\%
   \end{tabular*}
   \vfill}}
  \end{minipage}}}

%%  Lutetium
\newcommand{\ElemLu}{{%
  \begin{minipage}{67mm}%
  \vspace{1mm}

  {\Huge{\hspace{1mm} {\textbf{Lu}} \hfill \hfil{\textbf{71}} \hspace{1mm}}} %

  \vspace{6mm}

  {\Huge{\hfill {\Name{lutetium}} \hfill}}

  \vspace{6mm}

  {\Large{
  \begin{tabular*}{67mm}%
   {@{\hspace{5pt}}{r}@{\extracolsep{\fill}}r@{\extracolsep{\fill}}r}%
   %\multicolumn{3}{@{\hspace{1pt}}c}{ }\\%
   %\multicolumn{3}{c}{{\Huge{\BBlue{lutetium}}} }\\%
   % \multicolumn{3}{@{\hspace{1pt}}c}{ }\\%
   {\BRed{63314}}  & {\bf{54070}} &  {\bf{61290}} \\%
   {\BBlue{10870}} & 8846 & 8606 \\%
   {\BBlue{10349}} & 8710 & 10143 \\%
   {\BRed{9244}} & {\bf{7655}} & {\bf{9038}} \\%
   {\BBlue{1589}} & 1580 & 1630 \\%
   \multicolumn{3}{@{\hspace{1pt}}c}{ }\\%
   \end{tabular*}
   \vfill}}
  \end{minipage}}}

%%  Hafnium
\newcommand{\ElemHf}{{%
  \begin{minipage}{67mm}%
  \vspace{1mm}

  {\Huge{\hspace{1mm} {\textbf{Hf}} \hfill \hfil{\textbf{72}} \hspace{1mm}}} %

  \vspace{6mm}

  {\Huge{\hfill {\Name{hafnium}} \hfill}}

  \vspace{6mm}

  {\Large{
  \begin{tabular*}{67mm}%
   {@{\hspace{5pt}}{r}@{\extracolsep{\fill}}r@{\extracolsep{\fill}}r}%
   %\multicolumn{3}{@{\hspace{1pt}}c}{ }\\%
   %\multicolumn{3}{c}{{\Huge{\BBlue{hafnium}}} }\\%
   % \multicolumn{3}{@{\hspace{1pt}}c}{ }\\%
   {\BRed{65351}}  & {\bf{55790}} &  {\bf{63244}} \\%
   {\BBlue{11271}} & 9164 & 8906 \\%
   {\BBlue{10739}} & 9023 & 10519 \\%
   {\BRed{9561}} & {\bf{7899}} & {\bf{9341}} \\%
   {\BBlue{1662}} & 1646 & 1700 \\%
   \multicolumn{3}{@{\hspace{1pt}}c}{ }\\%
   \end{tabular*}
   \vfill}}
  \end{minipage}}}

%%  Tantalum
\newcommand{\ElemTa}{{%
  \begin{minipage}{67mm}%
  \vspace{1mm}

  {\Huge{\hspace{1mm} {\textbf{Ta}} \hfill \hfil{\textbf{73}} \hspace{1mm}}} %

  \vspace{6mm}

  {\Huge{\hfill {\Name{tantalum}} \hfill}}

  \vspace{6mm}

  {\Large{
  \begin{tabular*}{67mm}%
   {@{\hspace{5pt}}{r}@{\extracolsep{\fill}}r@{\extracolsep{\fill}}r}%
   %\multicolumn{3}{@{\hspace{1pt}}c}{ }\\%
   %\multicolumn{3}{c}{{\Huge{\BBlue{tantalum}}} }\\%
   % \multicolumn{3}{@{\hspace{1pt}}c}{ }\\%
   {\BRed{67416}}  & {\bf{57535}} &  {\bf{65222}} \\%
   {\BBlue{11682}} & 9488 & 9213 \\%
   {\BBlue{11136}} & 9343 & 10898 \\%
   {\BRed{9881}} & {\bf{8146}} & {\bf{9643}} \\%
   {\BBlue{1735}} & 1712 & 1770 \\%
   \multicolumn{3}{@{\hspace{1pt}}c}{ }\\%
   \end{tabular*}
   \vfill}}
  \end{minipage}}}

%%  Tungsten
\newcommand{\ElemW}{{%
  \begin{minipage}{67mm}%
  \vspace{1mm}

  {\Huge{\hspace{1mm} {\textbf{W}} \hfill \hfil{\textbf{74}} \hspace{1mm}}} %

  \vspace{6mm}

  {\Huge{\hfill {\Name{tungsten}} \hfill}}

  \vspace{6mm}

  {\Large{
  \begin{tabular*}{67mm}%
   {@{\hspace{5pt}}{r}@{\extracolsep{\fill}}r@{\extracolsep{\fill}}r}%
   %\multicolumn{3}{@{\hspace{1pt}}c}{ }\\%
   %\multicolumn{3}{c}{{\Huge{\BBlue{tungsten}}} }\\%
   % \multicolumn{3}{@{\hspace{1pt}}c}{ }\\%
   {\BRed{69525}}  & {\bf{59318}} &  {\bf{67244}} \\%
   {\BBlue{12100}} & 9819 & 9525 \\%
   {\BBlue{11544}} & 9672 & 11288 \\%
   {\BRed{10207}} & {\bf{8398}} & {\bf{9951}} \\%
   {\BBlue{1809}} & 1775 & 1838 \\%
   \multicolumn{3}{@{\hspace{1pt}}c}{ }\\%
   \end{tabular*}
   \vfill}}
  \end{minipage}}}

%%  Rhenium
\newcommand{\ElemRe}{{%
  \begin{minipage}{67mm}%
  \vspace{1mm}

  {\Huge{\hspace{1mm} {\textbf{Re}} \hfill \hfil{\textbf{75}} \hspace{1mm}}} %

  \vspace{6mm}

  {\Huge{\hfill {\Name{rhenium}} \hfill}}

  \vspace{6mm}

  {\Large{
  \begin{tabular*}{67mm}%
   {@{\hspace{5pt}}{r}@{\extracolsep{\fill}}r@{\extracolsep{\fill}}r}%
   %\multicolumn{3}{@{\hspace{1pt}}c}{ }\\%
   %\multicolumn{3}{c}{{\Huge{\BBlue{rhenium}}} }\\%
   % \multicolumn{3}{@{\hspace{1pt}}c}{ }\\%
   {\BRed{71676}}  & {\bf{61141}} &  {\bf{69309}} \\%
   {\BBlue{12527}} & 10160 & 9845 \\%
   {\BBlue{11959}} & 10010 & 11685 \\%
   {\BRed{10535}} & {\bf{8652}} & {\bf{10261}} \\%
   {\BBlue{1883}} & 1840 & 1906 \\%
   \multicolumn{3}{@{\hspace{1pt}}c}{ }\\%
   \end{tabular*}
   \vfill}}
  \end{minipage}}}

%%  Osmium
\newcommand{\ElemOs}{{%
  \begin{minipage}{67mm}%
  \vspace{1mm}

  {\Huge{\hspace{1mm} {\textbf{Os}} \hfill \hfil{\textbf{76}} \hspace{1mm}}} %

  \vspace{6mm}

  {\Huge{\hfill {\Name{osmium}} \hfill}}

  \vspace{6mm}

  {\Large{
  \begin{tabular*}{67mm}%
   {@{\hspace{5pt}}{r}@{\extracolsep{\fill}}r@{\extracolsep{\fill}}r}%
   %\multicolumn{3}{@{\hspace{1pt}}c}{ }\\%
   %\multicolumn{3}{c}{{\Huge{\BBlue{osmium}}} }\\%
   % \multicolumn{3}{@{\hspace{1pt}}c}{ }\\%
   {\BRed{73871}}  & {\bf{63000}} &  {\bf{71414}} \\%
   {\BBlue{12968}} & 10511 & 10176 \\%
   {\BBlue{12385}} & 10354 & 12092 \\%
   {\BRed{10871}} & {\bf{8911}} & {\bf{10578}} \\%
   {\BBlue{1960}} & 1907 & 1978 \\%
   \multicolumn{3}{@{\hspace{1pt}}c}{ }\\%
   \end{tabular*}
   \vfill}}
  \end{minipage}}}

%%  Iridium
\newcommand{\ElemIr}{{%
  \begin{minipage}{67mm}%
  \vspace{1mm}

  {\Huge{\hspace{1mm} {\textbf{Ir}} \hfill \hfil{\textbf{77}} \hspace{1mm}}} %

  \vspace{6mm}

  {\Huge{\hfill {\Name{iridium}} \hfill}}

  \vspace{6mm}

  {\Large{
  \begin{tabular*}{67mm}%
   {@{\hspace{5pt}}{r}@{\extracolsep{\fill}}r@{\extracolsep{\fill}}r}%
   %\multicolumn{3}{@{\hspace{1pt}}c}{ }\\%
   %\multicolumn{3}{c}{{\Huge{\BBlue{iridium}}} }\\%
   % \multicolumn{3}{@{\hspace{1pt}}c}{ }\\%
   {\BRed{76111}}  & {\bf{64896}} &  {\bf{73560}} \\%
   {\BBlue{13419}} & 10868 & 10510 \\%
   {\BBlue{12824}} & 10708 & 12512 \\%
   {\BRed{11215}} & {\bf{9175}} & {\bf{10903}} \\%
   {\BBlue{2040}} & 1976 & 2052 \\%
   \multicolumn{3}{@{\hspace{1pt}}c}{ }\\%
   \end{tabular*}
   \vfill}}
  \end{minipage}}}

%%  Platinum
\newcommand{\ElemPt}{{%
  \begin{minipage}{67mm}%
  \vspace{1mm}

  {\Huge{\hspace{1mm} {\textbf{Pt}} \hfill \hfil{\textbf{78}} \hspace{1mm}}} %

  \vspace{6mm}

  {\Huge{\hfill {\Name{platinum}} \hfill}}

  \vspace{6mm}

  {\Large{
  \begin{tabular*}{67mm}%
   {@{\hspace{5pt}}{r}@{\extracolsep{\fill}}r@{\extracolsep{\fill}}r}%
   %\multicolumn{3}{@{\hspace{1pt}}c}{ }\\%
   %\multicolumn{3}{c}{{\Huge{\BBlue{platinum}}} }\\%
   % \multicolumn{3}{@{\hspace{1pt}}c}{ }\\%
   {\BRed{78395}}  & {\bf{66831}} &  {\bf{75750}} \\%
   {\BBlue{13880}} & 11235 & 10853 \\%
   {\BBlue{13273}} & 11071 & 12941 \\%
   {\BRed{11564}} & {\bf{9442}} & {\bf{11232}} \\%
   {\BBlue{2122}} & 2048 & 2128 \\%
   \multicolumn{3}{@{\hspace{1pt}}c}{ }\\%
   \end{tabular*}
   \vfill}}
  \end{minipage}}}

%%  Gold
\newcommand{\ElemAu}{{%
  \begin{minipage}{67mm}%
  \vspace{1mm}

  {\Huge{\hspace{1mm} {\textbf{Au}} \hfill \hfil{\textbf{79}} \hspace{1mm}}} %

  \vspace{6mm}

  {\Huge{\hfill {\Name{gold}} \hfill}}

  \vspace{6mm}

  {\Large{
  \begin{tabular*}{67mm}%
   {@{\hspace{5pt}}{r}@{\extracolsep{\fill}}r@{\extracolsep{\fill}}r}%
   %\multicolumn{3}{@{\hspace{1pt}}c}{ }\\%
   %\multicolumn{3}{c}{{\Huge{\BBlue{gold}}} }\\%
   % \multicolumn{3}{@{\hspace{1pt}}c}{ }\\%
   {\BRed{80725}}  & {\bf{68806}} &  {\bf{77982}} \\%
   {\BBlue{14353}} & 11610 & 11205 \\%
   {\BBlue{13734}} & 11443 & 13381 \\%
   {\BRed{11919}} & {\bf{9713}} & {\bf{11566}} \\%
   {\BBlue{2206}} & 2118 & 2203 \\%
   \multicolumn{3}{@{\hspace{1pt}}c}{ }\\%
   \end{tabular*}
   \vfill}}
  \end{minipage}}}

%%  Mercury
\newcommand{\ElemHg}{{%
  \begin{minipage}{67mm}%
  \vspace{1mm}

  {\Huge{\hspace{1mm} {\textbf{Hg}} \hfill \hfil{\textbf{80}} \hspace{1mm}}} %

  \vspace{6mm}

  {\Huge{\hfill {\Name{mercury}} \hfill}}

  \vspace{6mm}

  {\Large{
  \begin{tabular*}{67mm}%
   {@{\hspace{5pt}}{r}@{\extracolsep{\fill}}r@{\extracolsep{\fill}}r}%
   %\multicolumn{3}{@{\hspace{1pt}}c}{ }\\%
   %\multicolumn{3}{c}{{\Huge{\BBlue{mercury}}} }\\%
   % \multicolumn{3}{@{\hspace{1pt}}c}{ }\\%
   {\BRed{83102}}  & {\bf{70818}} &  {\bf{80255}} \\%
   {\BBlue{14839}} & 11992 & 11560 \\%
   {\BBlue{14209}} & 11824 & 13831 \\%
   {\BRed{12284}} & {\bf{9989}} & {\bf{11906}} \\%
   {\BBlue{2295}} & 2191 & 2281 \\%
   \multicolumn{3}{@{\hspace{1pt}}c}{ }\\%
   \end{tabular*}
   \vfill}}
  \end{minipage}}}

%%  Thallium
\newcommand{\ElemTl}{{%
  \begin{minipage}{67mm}%
  \vspace{1mm}

  {\Huge{\hspace{1mm} {\textbf{Tl}} \hfill \hfil{\textbf{81}} \hspace{1mm}}} %

  \vspace{6mm}

  {\Huge{\hfill {\Name{thallium}} \hfill}}

  \vspace{6mm}

  {\Large{
  \begin{tabular*}{67mm}%
   {@{\hspace{5pt}}{r}@{\extracolsep{\fill}}r@{\extracolsep{\fill}}r}%
   %\multicolumn{3}{@{\hspace{1pt}}c}{ }\\%
   %\multicolumn{3}{c}{{\Huge{\BBlue{thallium}}} }\\%
   % \multicolumn{3}{@{\hspace{1pt}}c}{ }\\%
   {\BRed{85530}}  & {\bf{72872}} &  {\bf{82573}} \\%
   {\BBlue{15347}} & 12390 & 11931 \\%
   {\BBlue{14698}} & 12213 & 14292 \\%
   {\BRed{12658}} & {\bf{10269}} & {\bf{12252}} \\%
   {\BBlue{2389}} & 2267 & 2363 \\%
   \multicolumn{3}{@{\hspace{1pt}}c}{ }\\%
   \end{tabular*}
   \vfill}}
  \end{minipage}}}

%%  Lead
\newcommand{\ElemPb}{{%
  \begin{minipage}{67mm}%
  \vspace{1mm}

  {\Huge{\hspace{1mm} {\textbf{Pb}} \hfill \hfil{\textbf{82}} \hspace{1mm}}} %

  \vspace{6mm}

  {\Huge{\hfill {\Name{lead}} \hfill}}

  \vspace{6mm}

  {\Large{
  \begin{tabular*}{67mm}%
   {@{\hspace{5pt}}{r}@{\extracolsep{\fill}}r@{\extracolsep{\fill}}r}%
   %\multicolumn{3}{@{\hspace{1pt}}c}{ }\\%
   %\multicolumn{3}{c}{{\Huge{\BBlue{lead}}} }\\%
   % \multicolumn{3}{@{\hspace{1pt}}c}{ }\\%
   {\BRed{88005}}  & {\bf{74970}} &  {\bf{84939}} \\%
   {\BBlue{15861}} & 12795 & 12307 \\%
   {\BBlue{15200}} & 12614 & 14766 \\%
   {\BRed{13035}} & {\bf{10551}} & {\bf{12601}} \\%
   {\BBlue{2484}} & 2342 & 2444 \\%
   \multicolumn{3}{@{\hspace{1pt}}c}{ }\\%
   \end{tabular*}
   \vfill}}
  \end{minipage}}}

%%  Bismuth
\newcommand{\ElemBi}{{%
  \begin{minipage}{67mm}%
  \vspace{1mm}

  {\Huge{\hspace{1mm} {\textbf{Bi}} \hfill \hfil{\textbf{83}} \hspace{1mm}}} %

  \vspace{6mm}

  {\Huge{\hfill {\Name{bismuth}} \hfill}}

  \vspace{6mm}

  {\Large{
  \begin{tabular*}{67mm}%
   {@{\hspace{5pt}}{r}@{\extracolsep{\fill}}r@{\extracolsep{\fill}}r}%
   %\multicolumn{3}{@{\hspace{1pt}}c}{ }\\%
   %\multicolumn{3}{c}{{\Huge{\BBlue{bismuth}}} }\\%
   % \multicolumn{3}{@{\hspace{1pt}}c}{ }\\%
   {\BRed{90526}}  & {\bf{77107}} &  {\bf{87349}} \\%
   {\BBlue{16388}} & 13211 & 12692 \\%
   {\BBlue{15711}} & 13023 & 15247 \\%
   {\BRed{13419}} & {\bf{10839}} & {\bf{12955}} \\%
   {\BBlue{2580}} & 2418 & 2526 \\%
   \multicolumn{3}{@{\hspace{1pt}}c}{ }\\%
   \end{tabular*}
   \vfill}}
  \end{minipage}}}

%%  Polonium
\newcommand{\ElemPo}{{%
  \begin{minipage}{67mm}%
  \vspace{1mm}

  {\Huge{\hspace{1mm} {\textbf{Po}} \hfill \hfil{\textbf{84}} \hspace{1mm}}} %

  \vspace{6mm}

  {\Huge{\hfill {\Name{polonium}} \hfill}}

  \vspace{6mm}

  {\Large{
  \begin{tabular*}{67mm}%
   {@{\hspace{5pt}}{r}@{\extracolsep{\fill}}r@{\extracolsep{\fill}}r}%
   %\multicolumn{3}{@{\hspace{1pt}}c}{ }\\%
   %\multicolumn{3}{c}{{\Huge{\BBlue{polonium}}} }\\%
   % \multicolumn{3}{@{\hspace{1pt}}c}{ }\\%
   {\BRed{93105}}  & {\bf{79291}} &  {\bf{89803}} \\%
   {\BBlue{16939}} & 13637 & 13085 \\%
   {\BBlue{16244}} & 13446 & 15744 \\%
   {\BRed{13814}} & {\bf{11131}} & {\bf{13314}} \\%
   {\BBlue{2683}} & 2499 & 2614 \\%
   \multicolumn{3}{@{\hspace{1pt}}c}{ }\\%
   \end{tabular*}
   \vfill}}
  \end{minipage}}}

%%  Astatine
\newcommand{\ElemAt}{{%
  \begin{minipage}{67mm}%
  \vspace{1mm}

  {\Huge{\hspace{1mm} {\textbf{At}} \hfill \hfil{\textbf{85}} \hspace{1mm}}} %

  \vspace{6mm}

  {\Huge{\hfill {\Name{astatine}} \hfill}}

  \vspace{6mm}

  {\Large{
  \begin{tabular*}{67mm}%
   {@{\hspace{5pt}}{r}@{\extracolsep{\fill}}r@{\extracolsep{\fill}}r}%
   %\multicolumn{3}{@{\hspace{1pt}}c}{ }\\%
   %\multicolumn{3}{c}{{\Huge{\BBlue{astatine}}} }\\%
   % \multicolumn{3}{@{\hspace{1pt}}c}{ }\\%
   {\BRed{95730}}  & {\bf{81516}} &  {\bf{92304}} \\%
   {\BBlue{17493}} & 14067 & 13485 \\%
   {\BBlue{16785}} & 13876 & 16252 \\%
   {\BRed{14214}} & {\bf{11427}} & {\bf{13681}} \\%
   {\BBlue{2787}} & 2577 & 2699 \\%
   \multicolumn{3}{@{\hspace{1pt}}c}{ }\\%
   \end{tabular*}
   \vfill}}
  \end{minipage}}}

%%  Radon
\newcommand{\ElemRn}{{%
  \begin{minipage}{67mm}%
  \vspace{1mm}

  {\Huge{\hspace{1mm} {\textbf{Rn}} \hfill \hfil{\textbf{86}} \hspace{1mm}}} %

  \vspace{6mm}

  {\Huge{\hfill {\Name{radon}} \hfill}}

  \vspace{6mm}

  {\Large{
  \begin{tabular*}{67mm}%
   {@{\hspace{5pt}}{r}@{\extracolsep{\fill}}r@{\extracolsep{\fill}}r}%
   %\multicolumn{3}{@{\hspace{1pt}}c}{ }\\%
   %\multicolumn{3}{c}{{\Huge{\BBlue{radon}}} }\\%
   % \multicolumn{3}{@{\hspace{1pt}}c}{ }\\%
   {\BRed{98404}}  & {\bf{83785}} &  {\bf{94866}} \\%
   {\BBlue{18049}} & 14511 & 13890 \\%
   {\BBlue{17337}} & 14315 & 16770 \\%
   {\BRed{14619}} & {\bf{11727}} & {\bf{14052}} \\%
   {\BBlue{2892}} & 2654 & 2784 \\%
   \multicolumn{3}{@{\hspace{1pt}}c}{ }\\%
   \end{tabular*}
   \vfill}}
  \end{minipage}}}

%%  Francium
\newcommand{\ElemFr}{{%
  \begin{minipage}{67mm}%
  \vspace{1mm}

  {\Huge{\hspace{1mm} {\textbf{Fr}} \hfill \hfil{\textbf{87}} \hspace{1mm}}} %

  \vspace{6mm}

  {\Huge{\hfill {\Name{francium}} \hfill}}

  \vspace{6mm}

  {\Large{
  \begin{tabular*}{67mm}%
   {@{\hspace{5pt}}{r}@{\extracolsep{\fill}}r@{\extracolsep{\fill}}r}%
   %\multicolumn{3}{@{\hspace{1pt}}c}{ }\\%
   %\multicolumn{3}{c}{{\Huge{\BBlue{francium}}} }\\%
   % \multicolumn{3}{@{\hspace{1pt}}c}{ }\\%
   {\BRed{101137}}  & {\bf{86106}} &  {\bf{97474}} \\%
   {\BBlue{18639}} & 14976 & 14312 \\%
   {\BBlue{17907}} & 14771 & 17304 \\%
   {\BRed{15031}} & {\bf{12031}} & {\bf{14428}} \\%
   {\BBlue{3000}} & 2732 & 2868 \\%
   \multicolumn{3}{@{\hspace{1pt}}c}{ }\\%
   \end{tabular*}
   \vfill}}
  \end{minipage}}}

%%  Radium
\newcommand{\ElemRa}{{%
  \begin{minipage}{67mm}%
  \vspace{1mm}

  {\Huge{\hspace{1mm} {\textbf{Ra}} \hfill \hfil{\textbf{88}} \hspace{1mm}}} %

  \vspace{6mm}

  {\Huge{\hfill {\Name{radium}} \hfill}}

  \vspace{6mm}

  {\Large{
  \begin{tabular*}{67mm}%
   {@{\hspace{5pt}}{r}@{\extracolsep{\fill}}r@{\extracolsep{\fill}}r}%
   %\multicolumn{3}{@{\hspace{1pt}}c}{ }\\%
   %\multicolumn{3}{c}{{\Huge{\BBlue{radium}}} }\\%
   % \multicolumn{3}{@{\hspace{1pt}}c}{ }\\%
   {\BRed{103922}}  & {\bf{88478}} &  {\bf{100130}} \\%
   {\BBlue{19237}} & 15445 & 14747 \\%
   {\BBlue{18484}} & 15236 & 17848 \\%
   {\BRed{15444}} & {\bf{12339}} & {\bf{14808}} \\%
   {\BBlue{3105}} & 2806 & 2949 \\%
   \multicolumn{3}{@{\hspace{1pt}}c}{ }\\%
   \end{tabular*}
   \vfill}}
  \end{minipage}}}

%%  Actinium
\newcommand{\ElemAc}{{%
  \begin{minipage}{67mm}%
  \vspace{1mm}

  {\Huge{\hspace{1mm} {\textbf{Ac}} \hfill \hfil{\textbf{89}} \hspace{1mm}}} %

  \vspace{6mm}

  {\Huge{\hfill {\Name{actinium}} \hfill}}

  \vspace{6mm}

  {\Large{
  \begin{tabular*}{67mm}%
   {@{\hspace{5pt}}{r}@{\extracolsep{\fill}}r@{\extracolsep{\fill}}r}%
   %\multicolumn{3}{@{\hspace{1pt}}c}{ }\\%
   %\multicolumn{3}{c}{{\Huge{\BBlue{actinium}}} }\\%
   % \multicolumn{3}{@{\hspace{1pt}}c}{ }\\%
   {\BRed{106755}}  & {\bf{90884}} &  {\bf{102846}} \\%
   {\BBlue{19840}} & 15931 & 15184 \\%
   {\BBlue{19083}} & 15713 & 18408 \\%
   {\BRed{15871}} & {\bf{12652}} & {\bf{15196}} \\%
   {\BBlue{3219}} & 2900 & 3051 \\%
   \multicolumn{3}{@{\hspace{1pt}}c}{ }\\%
   \end{tabular*}
   \vfill}}
  \end{minipage}}}

%%  Thorium
\newcommand{\ElemTh}{{%
  \begin{minipage}{67mm}%
  \vspace{1mm}

  {\Huge{\hspace{1mm} {\textbf{Th}} \hfill \hfil{\textbf{90}} \hspace{1mm}}} %

  \vspace{6mm}

  {\Huge{\hfill {\Name{thorium}} \hfill}}

  \vspace{6mm}

  {\Large{
  \begin{tabular*}{67mm}%
   {@{\hspace{5pt}}{r}@{\extracolsep{\fill}}r@{\extracolsep{\fill}}r}%
   %\multicolumn{3}{@{\hspace{1pt}}c}{ }\\%
   %\multicolumn{3}{c}{{\Huge{\BBlue{thorium}}} }\\%
   % \multicolumn{3}{@{\hspace{1pt}}c}{ }\\%
   {\BRed{109651}}  & {\bf{93351}} &  {\bf{105605}} \\%
   {\BBlue{20472}} & 16426 & 15642 \\%
   {\BBlue{19693}} & 16202 & 18981 \\%
   {\BRed{16300}} & {\bf{12968}} & {\bf{15588}} \\%
   {\BBlue{3332}} & 2990 & 3149 \\%
   \multicolumn{3}{@{\hspace{1pt}}c}{ }\\%
   \end{tabular*}
   \vfill}}
  \end{minipage}}}

%%  Protactinium
\newcommand{\ElemPa}{{%
  \begin{minipage}{67mm}%
  \vspace{1mm}

  {\Huge{\hspace{1mm} {\textbf{Pa}} \hfill \hfil{\textbf{91}} \hspace{1mm}}} %

  \vspace{6mm}

  {\Huge{\hfill {\Name{protactinium}} \hfill}}

  \vspace{6mm}

  {\Large{
  \begin{tabular*}{67mm}%
   {@{\hspace{5pt}}{r}@{\extracolsep{\fill}}r@{\extracolsep{\fill}}r}%
   %\multicolumn{3}{@{\hspace{1pt}}c}{ }\\%
   %\multicolumn{3}{c}{{\Huge{\BBlue{protactinium}}} }\\%
   % \multicolumn{3}{@{\hspace{1pt}}c}{ }\\%
   {\BRed{112601}}  & {\bf{95868}} &  {\bf{108427}} \\%
   {\BBlue{21105}} & 16931 & 16104 \\%
   {\BBlue{20314}} & 16703 & 19571 \\%
   {\BRed{16733}} & {\bf{13291}} & {\bf{15990}} \\%
   {\BBlue{3442}} & 3071 & 3240 \\%
   \multicolumn{3}{@{\hspace{1pt}}c}{ }\\%
   \end{tabular*}
   \vfill}}
  \end{minipage}}}

%%  Uranium
\newcommand{\ElemU}{{%
  \begin{minipage}{67mm}%
  \vspace{1mm}

  {\Huge{\hspace{1mm} {\textbf{U}} \hfill \hfil{\textbf{92}} \hspace{1mm}}} %

  \vspace{6mm}

  {\Huge{\hfill {\Name{uranium}} \hfill}}

  \vspace{6mm}

  {\Large{
  \begin{tabular*}{67mm}%
   {@{\hspace{5pt}}{r}@{\extracolsep{\fill}}r@{\extracolsep{\fill}}r}%
   %\multicolumn{3}{@{\hspace{1pt}}c}{ }\\%
   %\multicolumn{3}{c}{{\Huge{\BBlue{uranium}}} }\\%
   % \multicolumn{3}{@{\hspace{1pt}}c}{ }\\%
   {\BRed{115606}}  & {\bf{98440}} &  {\bf{111303}} \\%
   {\BBlue{21757}} & 17454 & 16575 \\%
   {\BBlue{20948}} & 17220 & 20170 \\%
   {\BRed{17166}} & {\bf{13614}} & {\bf{16388}} \\%
   {\BBlue{3552}} & 3164 & 3340 \\%
   \multicolumn{3}{@{\hspace{1pt}}c}{ }\\%
   \end{tabular*}
   \vfill}}
  \end{minipage}}}

%%  Neptunium
\newcommand{\ElemNp}{{%
  \begin{minipage}{67mm}%
  \vspace{1mm}

  {\Huge{\hspace{1mm} {\textbf{Np}} \hfill \hfil{\textbf{93}} \hspace{1mm}}} %

  \vspace{6mm}

  {\Huge{\hfill {\Name{neptunium}} \hfill}}

  \vspace{6mm}

  {\Large{
  \begin{tabular*}{67mm}%
   {@{\hspace{5pt}}{r}@{\extracolsep{\fill}}r@{\extracolsep{\fill}}r}%
   %\multicolumn{3}{@{\hspace{1pt}}c}{ }\\%
   %\multicolumn{3}{c}{{\Huge{\BBlue{neptunium}}} }\\%
   % \multicolumn{3}{@{\hspace{1pt}}c}{ }\\%
   {\BRed{118669}}  & {\bf{101059}} &  {\bf{114234}} \\%
   {\BBlue{22427}} & 17992 & 17061 \\%
   {\BBlue{21600}} & 17751 & 20784 \\%
   {\BRed{17610}} & {\bf{13946}} & {\bf{16794}} \\%
   {\BBlue{3664}} & 3250 & 3435 \\%
   \multicolumn{3}{@{\hspace{1pt}}c}{ }\\%
   \end{tabular*}
   \vfill}}
  \end{minipage}}}

%%  Plutonium
\newcommand{\ElemPu}{{%
  \begin{minipage}{67mm}%
  \vspace{1mm}

  {\Huge{\hspace{1mm} {\textbf{Pu}} \hfill \hfil{\textbf{94}} \hspace{1mm}}} %

  \vspace{6mm}

  {\Huge{\hfill {\Name{plutonium}} \hfill}}

  \vspace{6mm}

  {\Large{
  \begin{tabular*}{67mm}%
   {@{\hspace{5pt}}{r}@{\extracolsep{\fill}}r@{\extracolsep{\fill}}r}%
   %\multicolumn{3}{@{\hspace{1pt}}c}{ }\\%
   %\multicolumn{3}{c}{{\Huge{\BBlue{plutonium}}} }\\%
   % \multicolumn{3}{@{\hspace{1pt}}c}{ }\\%
   {\BRed{121791}}  & {\bf{103734}} &  {\bf{117228}} \\%
   {\BBlue{23104}} & 18541 & 17557 \\%
   {\BBlue{22266}} & 18296 & 21420 \\%
   {\BRed{18057}} & {\bf{14282}} & {\bf{17211}} \\%
   {\BBlue{3775}} & 3339 & 3534 \\%
   \multicolumn{3}{@{\hspace{1pt}}c}{ }\\%
   \end{tabular*}
   \vfill}}
  \end{minipage}}}

%%  Americium
\newcommand{\ElemAm}{{%
  \begin{minipage}{67mm}%
  \vspace{1mm}

  {\Huge{\hspace{1mm} {\textbf{Am}} \hfill \hfil{\textbf{95}} \hspace{1mm}}} %

  \vspace{6mm}

  {\Huge{\hfill {\Name{americium}} \hfill}}

  \vspace{6mm}

  {\Large{
  \begin{tabular*}{67mm}%
   {@{\hspace{5pt}}{r}@{\extracolsep{\fill}}r@{\extracolsep{\fill}}r}%
   %\multicolumn{3}{@{\hspace{1pt}}c}{ }\\%
   %\multicolumn{3}{c}{{\Huge{\BBlue{americium}}} }\\%
   % \multicolumn{3}{@{\hspace{1pt}}c}{ }\\%
   {\BRed{124982}}  & {\bf{106472}} &  {\bf{120284}} \\%
   {\BBlue{23808}} & 19110 & 18069 \\%
   {\BBlue{22952}} & 18856 & 22072 \\%
   {\BRed{18510}} & {\bf{14620}} & {\bf{17630}} \\%
   {\BBlue{3890}} & 3429 & 3635 \\%
   \multicolumn{3}{@{\hspace{1pt}}c}{ }\\%
   \end{tabular*}
   \vfill}}
  \end{minipage}}}

%%  Curium
\newcommand{\ElemCm}{{%
  \begin{minipage}{67mm}%
  \vspace{1mm}

  {\Huge{\hspace{1mm} {\textbf{Cm}} \hfill \hfil{\textbf{96}} \hspace{1mm}}} %

  \vspace{6mm}

  {\Huge{\hfill {\Name{curium}} \hfill}}

  \vspace{6mm}

  {\Large{
  \begin{tabular*}{67mm}%
   {@{\hspace{5pt}}{r}@{\extracolsep{\fill}}r@{\extracolsep{\fill}}r}%
   %\multicolumn{3}{@{\hspace{1pt}}c}{ }\\%
   %\multicolumn{3}{c}{{\Huge{\BBlue{curium}}} }\\%
   % \multicolumn{3}{@{\hspace{1pt}}c}{ }\\%
   {\BRed{128241}}  & {\bf{109271}} &  {\bf{123403}} \\%
   {\BBlue{24526}} & 19688 & 18589 \\%
   {\BBlue{23651}} & 19427 & 22735 \\%
   {\BRed{18970}} & {\bf{14961}} & {\bf{18054}} \\%
   {\BBlue{4009}} & 3525 & 3740 \\%
   \multicolumn{3}{@{\hspace{1pt}}c}{ }\\%
   \end{tabular*}
   \vfill}}
  \end{minipage}}}

%%  Berkelium
\newcommand{\ElemBk}{{%
  \begin{minipage}{67mm}%
  \vspace{1mm}

  {\Huge{\hspace{1mm} {\textbf{Bk}} \hfill \hfil{\textbf{97}} \hspace{1mm}}} %

  \vspace{6mm}

  {\Huge{\hfill {\Name{berkelium}} \hfill}}

  \vspace{6mm}

  {\Large{
  \begin{tabular*}{67mm}%
   {@{\hspace{5pt}}{r}@{\extracolsep{\fill}}r@{\extracolsep{\fill}}r}%
   %\multicolumn{3}{@{\hspace{1pt}}c}{ }\\%
   %\multicolumn{3}{c}{{\Huge{\BBlue{berkelium}}} }\\%
   % \multicolumn{3}{@{\hspace{1pt}}c}{ }\\%
   {\BRed{131556}}  & {\bf{112121}} &  {\bf{126580}} \\%
   {\BBlue{25256}} & 20280 & 19118 \\%
   {\BBlue{24371}} & 20018 & 23416 \\%
   {\BRed{19435}} & {\bf{15308}} & {\bf{18480}} \\%
   {\BBlue{4127}} & 3616 & 3842 \\%
   \multicolumn{3}{@{\hspace{1pt}}c}{ }\\%
   \end{tabular*}
   \vfill}}
  \end{minipage}}}

%%  Californium
\newcommand{\ElemCf}{{%
  \begin{minipage}{67mm}%
  \vspace{1mm}

  {\Huge{\hspace{1mm} {\textbf{Cf}} \hfill \hfil{\textbf{98}} \hspace{1mm}}} %

  \vspace{6mm}

  {\Huge{\hfill {\Name{californium}} \hfill}}

  \vspace{6mm}

  {\Large{
  \begin{tabular*}{67mm}%
   {@{\hspace{5pt}}{r}@{\extracolsep{\fill}}r@{\extracolsep{\fill}}r}%
   %\multicolumn{3}{@{\hspace{1pt}}c}{ }\\%
   %\multicolumn{3}{c}{{\Huge{\BBlue{californium}}} }\\%
   % \multicolumn{3}{@{\hspace{1pt}}c}{ }\\%
   {\BRed{134939}}  & {\bf{115032}} &  {\bf{129823}} \\%
   {\BBlue{26010}} & 20894 & 19665 \\%
   {\BBlue{25108}} & 20624 & 24117 \\%
   {\BRed{19907}} & {\bf{15660}} & {\bf{18916}} \\%
   {\BBlue{4247}} & 3709 & 3946 \\%
   \multicolumn{3}{@{\hspace{1pt}}c}{ }\\%
   \end{tabular*}
   \vfill}}
  \end{minipage}}}

%%  Einsteinium
\newcommand{\ElemEs}{{%
  \begin{minipage}{67mm}%
  \vspace{1mm}

  {\Huge{\hspace{1mm} {\textbf{Es}} \hfill \hfil{\textbf{99}} \hspace{1mm}}} %

  \vspace{6mm}

  {\Huge{\hfill {\Name{einsteinium}} \hfill}}

  \vspace{6mm}

  {\Large{
  \begin{tabular*}{67mm}%
   {@{\hspace{5pt}}{r}@{\extracolsep{\fill}}r@{\extracolsep{\fill}}r}%
   %\multicolumn{3}{@{\hspace{1pt}}c}{ }\\%
   %\multicolumn{3}{c}{{\Huge{\BBlue{einsteinium}}} }\\%
   % \multicolumn{3}{@{\hspace{1pt}}c}{ }\\%
   {\BRed{}}  & {\bf{}} &  {\bf{}} \\%
   {\BBlue{}} &  &  \\%
   {\BBlue{}} &  &  \\%
   {\BRed{}} & {\bf{}} & {\bf{}} \\%
   {\BBlue{}} &  &  \\%
   \multicolumn{3}{@{\hspace{1pt}}c}{ }\\%
   \end{tabular*}
   \vfill}}
  \end{minipage}}}

%%  Fermium
\newcommand{\ElemFm}{{%
  \begin{minipage}{67mm}%
  \vspace{1mm}

  {\Huge{\hspace{1mm} {\textbf{Fm}} \hfill \hfil{\textbf{100}} \hspace{1mm}}} %

  \vspace{6mm}

  {\Huge{\hfill {\Name{fermium}} \hfill}}

  \vspace{6mm}

  {\Large{
  \begin{tabular*}{67mm}%
   {@{\hspace{5pt}}{r}@{\extracolsep{\fill}}r@{\extracolsep{\fill}}r}%
   %\multicolumn{3}{@{\hspace{1pt}}c}{ }\\%
   %\multicolumn{3}{c}{{\Huge{\BBlue{fermium}}} }\\%
   % \multicolumn{3}{@{\hspace{1pt}}c}{ }\\%
   {\BRed{}}  & {\bf{}} &  {\bf{}} \\%
   {\BBlue{}} &  &  \\%
   {\BBlue{}} &  &  \\%
   {\BRed{}} & {\bf{}} & {\bf{}} \\%
   {\BBlue{}} &  &  \\%
   \multicolumn{3}{@{\hspace{1pt}}c}{ }\\%
   \end{tabular*}
   \vfill}}
  \end{minipage}}}

%%  Mendelevium
\newcommand{\ElemMd}{{%
  \begin{minipage}{67mm}%
  \vspace{1mm}

  {\Huge{\hspace{1mm} {\textbf{Md}} \hfill \hfil{\textbf{101}} \hspace{1mm}}} %

  \vspace{6mm}

  {\Huge{\hfill {\Name{mendelevium}} \hfill}}

  \vspace{6mm}

  {\Large{
  \begin{tabular*}{67mm}%
   {@{\hspace{5pt}}{r}@{\extracolsep{\fill}}r@{\extracolsep{\fill}}r}%
   %\multicolumn{3}{@{\hspace{1pt}}c}{ }\\%
   %\multicolumn{3}{c}{{\Huge{\BBlue{mendelevium}}} }\\%
   % \multicolumn{3}{@{\hspace{1pt}}c}{ }\\%
   {\BRed{}}  & {\bf{}} &  {\bf{}} \\%
   {\BBlue{}} &  &  \\%
   {\BBlue{}} &  &  \\%
   {\BRed{}} & {\bf{}} & {\bf{}} \\%
   {\BBlue{}} &  &  \\%
   \multicolumn{3}{@{\hspace{1pt}}c}{ }\\%
   \end{tabular*}
   \vfill}}
  \end{minipage}}}

%%  Nobelium
\newcommand{\ElemNo}{{%
  \begin{minipage}{67mm}%
  \vspace{1mm}

  {\Huge{\hspace{1mm} {\textbf{No}} \hfill \hfil{\textbf{102}} \hspace{1mm}}} %

  \vspace{6mm}

  {\Huge{\hfill {\Name{nobelium}} \hfill}}

  \vspace{6mm}

  {\Large{
  \begin{tabular*}{67mm}%
   {@{\hspace{5pt}}{r}@{\extracolsep{\fill}}r@{\extracolsep{\fill}}r}%
   %\multicolumn{3}{@{\hspace{1pt}}c}{ }\\%
   %\multicolumn{3}{c}{{\Huge{\BBlue{nobelium}}} }\\%
   % \multicolumn{3}{@{\hspace{1pt}}c}{ }\\%
   {\BRed{}}  & {\bf{}} &  {\bf{}} \\%
   {\BBlue{}} &  &  \\%
   {\BBlue{}} &  &  \\%
   {\BRed{}} & {\bf{}} & {\bf{}} \\%
   {\BBlue{}} &  &  \\%
   \multicolumn{3}{@{\hspace{1pt}}c}{ }\\%
   \end{tabular*}
   \vfill}}
  \end{minipage}}}

%%  Lawrencium
\newcommand{\ElemLr}{{%
  \begin{minipage}{67mm}%
  \vspace{1mm}

  {\Huge{\hspace{1mm} {\textbf{Lr}} \hfill \hfil{\textbf{103}} \hspace{1mm}}} %

  \vspace{6mm}

  {\Huge{\hfill {\Name{lawrencium}} \hfill}}

  \vspace{6mm}

  {\Large{
  \begin{tabular*}{67mm}%
   {@{\hspace{5pt}}{r}@{\extracolsep{\fill}}r@{\extracolsep{\fill}}r}%
   %\multicolumn{3}{@{\hspace{1pt}}c}{ }\\%
   %\multicolumn{3}{c}{{\Huge{\BBlue{lawrencium}}} }\\%
   % \multicolumn{3}{@{\hspace{1pt}}c}{ }\\%
   {\BRed{}}  & {\bf{}} &  {\bf{}} \\%
   {\BBlue{}} &  &  \\%
   {\BBlue{}} &  &  \\%
   {\BRed{}} & {\bf{}} & {\bf{}} \\%
   {\BBlue{}} &  &  \\%
   \multicolumn{3}{@{\hspace{1pt}}c}{ }\\%
   \end{tabular*}
   \vfill}}
  \end{minipage}}}

%% Key
\newcommand{\ElemKey}{{%
  \begin{minipage}{67mm}%
  \vspace{1mm}

  {\Huge{\hspace{1mm} {\textbf{Symbol}} \hfill \hfil{\textbf{Z}} \hspace{1mm}}} %

  \vspace{6mm}

  {\Huge{\hfill {\Name{name}} \hfill}}

  \vspace{6mm}

  {\Large{
  \begin{tabular*}{67mm}%
   {@{\hspace{5pt}}{r}@{\extracolsep{\fill}}r@{\extracolsep{\fill}}r}%
   %\multicolumn{3}{@{\hspace{1pt}}c}{ }\\%
   %\multicolumn{3}{c}{{\Huge{\BBlue{name}}} }\\%
   % \multicolumn{3}{@{\hspace{1pt}}c}{ }\\%
   {\BRed{$\mathbf{K}$ edge}}  & {\bf{$\mathbf{K_{\alpha_1}}$}} &  {\bf{$\mathbf{K_{\beta_1}}$}} \\%
   {\BBlue{$\mathrm{L_{\rm I}}$ edge}} & $\mathrm{L_{\beta_3}}$ & $\mathrm{L_{\beta_4}}$ \\%
   {\BBlue{$\mathrm{L_{\rm II}}$ edge}} & $\mathrm{L_{\beta_1}}$ & $\mathrm{L_{\gamma_1}}$ \\%
   {\BRed{$\mathbf{L_{\rm III}}$ edge}} & {\bf{$\mathbf{L_{\alpha_1}}$}} & {\bf{$\mathbf{L_{\beta_2}}$}} \\%
   {\BBlue{$\mathrm{M_{\rm V}}$ edge}} & $\mathrm{M_{\alpha}}$ & $\mathrm{M_{\beta}}$ \\%
   \multicolumn{3}{@{\hspace{1pt}}c}{ }\\%
   \end{tabular*}
   \vfill}}
  \end{minipage}}}



%% Group 1 - IA
  \node[name=H, Element] {\ElemH};
  \node[name=Li, below of=H,  Element] {\ElemLi};
  \node[name=Na, below of=Li, Element] {\ElemNa};
  \node[name=K, below of=Na, Element] {\ElemK};
  \node[name=Rb, below of=K, Element] {\ElemRb};
  \node[name=Cs, below of=Rb, Element] {\ElemCs};
  \node[name=Fr, below of=Cs, Element] {\ElemFr};

%% Group 2 - IIA
  \node[name=Be, right of=Li, Element] {\ElemBe};
  \node[name=Mg, below of=Be, Element] {\ElemMg};
  \node[name=Ca, below of=Mg, Element] {\ElemCa};
  \node[name=Sr, below of=Ca, Element] {\ElemSr};
  \node[name=Ba, below of=Sr, Element] {\ElemBa};
  \node[name=Ra, below of=Ba, Element] {\ElemRa};

%% Group 3 - IIIB
  \node[name=Sc, right of=Ca, Element] {\ElemSc};
  \node[name=Y, below of=Sc, Element] {\ElemY};
  \node[name=La, below of=Y, Element] {\ElemLa};
  \node[name=Ac, below of=La, Element]{\ElemAc};

%% Group 4 - IVB
  \node[name=Ti, right of=Sc, Element] {\ElemTi};
  \node[name=Zr, below of=Ti, Element] {\ElemZr};
  \node[name=Hf, below of=Zr, Element] {\ElemHf};

%% Group 5 - VB
  \node[name=V, right of=Ti, Element] {\ElemV};
  \node[name=Nb, below of=V, Element] {\ElemNb};
  \node[name=Ta, below of=Nb, Element] {\ElemTa};

%% Group 6 - VIB
  \node[name=Cr, right of=V, Element] {\ElemCr};
  \node[name=Mo, below of=Cr, Element] {\ElemMo};
  \node[name=W, below of=Mo, Element] {\ElemW};

%% Group 7 - VIIB
  \node[name=Mn, right of=Cr, Element] {\ElemMn};
  \node[name=Tc, below of=Mn, Element] {\ElemTc};
  \node[name=Re, below of=Tc, Element] {\ElemRe};

%% Group 8 - VIIIB
  \node[name=Fe, right of=Mn, Element] {\ElemFe};
  \node[name=Ru, below of=Fe, Element] {\ElemRu};
  \node[name=Os, below of=Ru, Element] {\ElemOs};

%% Group 9 - VIIIB
  \node[name=Co, right of=Fe, Element] {\ElemCo};
  \node[name=Rh, below of=Co, Element] {\ElemRh};
  \node[name=Ir, below of=Rh, Element] {\ElemIr};

%% Group 10 - VIIIB
  \node[name=Ni, right of=Co, Element] {\ElemNi};
  \node[name=Pd, below of=Ni, Element] {\ElemPd};
  \node[name=Pt, below of=Pd, Element] {\ElemPt};

%% Group 11 - IB
  \node[name=Cu, right of=Ni, Element] {\ElemCu};
  \node[name=Ag, below of=Cu, Element] {\ElemAg};
  \node[name=Au, below of=Ag, Element] {\ElemAu};

%% Group 12 - IIB
  \node[name=Zn, right of=Cu, Element] {\ElemZn};
  \node[name=Cd, below of=Zn, Element] {\ElemCd};
  \node[name=Hg, below of=Cd, Element] {\ElemHg};

%% Group 13 - IIIA
  \node[name=Ga, right of=Zn, Element] {\ElemGa};
  \node[name=Al, above of=Ga, Element] {\ElemAl};
  \node[name=B, above of=Al, Element] {\ElemB};
  \node[name=In, below of=Ga, Element] {\ElemIn};
  \node[name=Tl, below of=In, Element] {\ElemTl};

%% Group 14 - IVA
  \node[name=C, right of=B, Element] {\ElemC};
  \node[name=Si, below of=C, Element] {\ElemSi};
  \node[name=Ge, below of=Si, Element] {\ElemGe};
  \node[name=Sn, below of=Ge, Element] {\ElemSn};
  \node[name=Pb, below of=Sn, Element] {\ElemPb};

%% Group 15 - VA
  \node[name=N, right of=C, Element] {\ElemN};
  \node[name=P, below of=N, Element] {\ElemP};
  \node[name=As, below of=P, Element] {\ElemAs};
  \node[name=Sb, below of=As, Element] {\ElemSb};
  \node[name=Bi, below of=Sb, Element] {\ElemBi};

%% Group 16 - VIA
  \node[name=O, right of=N, Element] {\ElemO};
  \node[name=S, below of=O, Element] {\ElemS};
  \node[name=Se, below of=S, Element] {\ElemSe};
  \node[name=Te, below of=Se, Element] {\ElemTe};
  \node[name=Po, below of=Te, Element] {\ElemPo};

%% Group 17 - VIIA
  \node[name=F, right of=O, Element] {\ElemF};
  \node[name=Cl, below of=F, Element] {\ElemCl};
  \node[name=Br, below of=Cl, Element] {\ElemBr};
  \node[name=I, below of=Br, Element] {\ElemI};
  \node[name=At, below of=I, Element] {\ElemAt};

%% Group 18 - VIIIA
  \node[name=Ne, right of=F, Element] {\ElemNe};
  \node[name=He, above of=Ne, Element] {\ElemHe};
  \node[name=Ar, below of=Ne, Element] {\ElemAr};
  \node[name=Kr, below of=Ar, Element] {\ElemKr};
  \node[name=Xe, below of=Kr, Element] {\ElemXe};
  \node[name=Rn, below of=Xe, Element] {\ElemRn};


%% Lanthanide
  \node[name=Ce, below of=Hf, Element,
            xshift=35mm, yshift=-35mm] {\ElemCe};
  \node[name=Pr, right of=Ce, Element] {\ElemPr};
  \node[name=Nd, right of=Pr, Element] {\ElemNd};
  \node[name=Pm, right of=Nd, Element] {\ElemPm};
  \node[name=Sm, right of=Pm, Element] {\ElemSm};
  \node[name=Eu, right of=Sm, Element] {\ElemEu};
  \node[name=Gd, right of=Eu, Element] {\ElemGd};
  \node[name=Tb, right of=Gd, Element] {\ElemTb};
  \node[name=Dy, right of=Tb, Element] {\ElemDy};
  \node[name=Ho, right of=Dy, Element] {\ElemHo};
  \node[name=Er, right of=Ho, Element] {\ElemEr};
  \node[name=Tm, right of=Er, Element] {\ElemTm};
  \node[name=Yb, right of=Tm, Element] {\ElemYb};
  \node[name=Lu, right of=Yb, Element] {\ElemLu};

%% Actinide
  \node[name=Th, below of=Ce, Element] {\ElemTh};
  \node[name=Pa, right of=Th, Element] {\ElemPa};
  \node[name=U, right of=Pa, Element] {\ElemU};
  \node[name=Np, right of=U, Element] {\ElemNp};
  \node[name=Pu, right of=Np, Element] {\ElemPu};
  \node[name=Am, right of=Pu, Element] {\ElemAm};
  \node[name=Cm, right of=Am, Element] {\ElemCm};
  \node[name=Bk, right of=Cm, Element] {\ElemBk};
  \node[name=Cf, right of=Bk, Element] {\ElemCf};
  \node[name=Es, right of=Cf, Element] {\ElemEs};
  \node[name=Fm, right of=Es, Element] {\ElemFm};
  \node[name=Md, right of=Fm, Element] {\ElemMd};
  \node[name=No, right of=Md, Element] {\ElemNo};
  \node[name=Lr, right of=No, Element] {\ElemLr};

% Title, subtitle, key
  \node at ($(H.west -| Co.north) + (0mm, 15mm)$) [name=diagramTitle, TitleLabel]
    {X-ray Absorption and Emission Energies of the Elements};

  \node at ($(Na.west -| V.north) + (-7em, 7em)$) [name=diagramTitle, SubTitleLabel]
  {\small{\textsf{
      \begin{tabular*}{5mm}{l}
        Atomic Data and Energies from \\%
        W. T. Elam, B. D. Ravel and  J. R. Sieber, \hfil\\%
        {\emph{Radiation Physics and Chemistry}}
        {\bf{63}}, pp 121-128 (2002)  \\%
        \noalign{\smallskip} \\%
        Common oxidation states from wikipedia.org, after\\%
        N. N. Greenwood and A. Earnshaw, \\
      {\emph{Chemistry of the Elements}}, 2nd  ed. (1997).\\%
        \noalign{\smallskip} \\%
        {All energies in eV.} \hfil\\%
        {Emission line strengths are approximate, and vary with element.} \hfil\\%
      \end{tabular*}
    }}};

 \node at ($(Mg.west -| Ti.north) + (0em,7em)$) [name=elementKey, Element]{\ElemKey};

%% add some extra space around table
  \node at ($(Lr.south -| Rn.east) + (2em, -2em)$)  [name=Space1, Space]{ };
  \node at ($(H.north  -|  H.west) - (2em, -2em)$)  [name=Space2, Space]{ };


\pgfdeclareimage[height=35mm]{Barkla}{images/Charles_Barkla}
\pgfdeclareimage[width=20mm]{qrcode}{images/xraytable_qr}

   \node at ($(Th.south -| Fr.west) + (35mm, 7mm)$) [name=diagramTitle,
   SubTitleLabel]{
     {\small{\textsf{
           \begin{tabular}{c}
           \pgfbox[center,bottom]{\pgfuseimage{Barkla}}\\
           Charles G. Barkla\\
         \end{tabular}
       }}}
   };

   \node at ($(Ce.south -| Ac.west) + (15mm, 5mm)$) [name=diagramTitle,
   SubTitleLabel]{
     {\small{\textsf{
           \begin{tabular}{l}
             This Periodic Table is freely available at:\\
             http://xafs.org/Databases/XrayTable \hfil \\%
             \noalign{\smallskip}
             Version 2, 26-Mar-2013\\
             \noalign{\smallskip}
             \hspace{25mm} \pgfbox[center,top]{\pgfuseimage{qrcode}}\\
           \end{tabular}
           }}}};


%% add emission line table
  %%
%% emission line table

\tikzset{glevel/.style={line width=0.8mm, draw=black!40}}
\tikzset{blevel/.style={line width=0.8mm, draw=blue!80}}
\tikzset{rlevel/.style={line width=0.8mm, draw=red!80}}
\tikzset{xray/.style={->,>=triangle 45, line width=1mm, font={\sffamily\Large}}}
\tikzset{seig/.style={fill=white, font={\sffamily\huge}}}
\tikzset{edge/.style={left, font={\sffamily\Large}}}
\tikzset{orbital/.style={right, font={\sffamily\Large}}}
\tikzset{xjoin/.style={line width=1mm}}
  
\begin{scope}[shift={(-18.5:52cm)}]
  {\Huge{
       \draw[rlevel] (0,   0mm) node[edge]{$K$}    --  ++(29, 0) node[orbital]  {$1s$};
       \draw[blevel] (0,  40mm) node[edge]{$L_1$}  --  ++(29, 0) node[orbital] {$2s$};
       \draw[blevel] (0,  45mm) node[edge]{$L_2$}  --  ++(29, 0) node[orbital] {$2p_{1/2}$};
       \draw[rlevel] (0,  50mm) node[edge]{$L_3$}  --  ++(29, 0) node[orbital] {$2p_{3/2}$};

       \draw[glevel] (0,  80mm) node[edge] {$M_1$} --  ++(29, 0) node[orbital] {$3s$};
       \draw[glevel] (0,  85mm) node[edge] {}      --  ++(29, 0) node[orbital] {$3p_{1/2}$};
       \draw[glevel] (0,  90mm) node[edge] {$M_3$} --  ++(29, 0) node[orbital] {$3p_{3/2}$};
       \draw[glevel] (0,  95mm) node[edge] {}      --  ++(29, 0) node[orbital] {$3d_{3/2}$};
       \draw[rlevel] (0, 100mm) node[edge] {$M_5$} --  ++(29, 0) node[orbital] {$3d_{5/2}$};

       \draw[glevel] (0, 120mm) node[edge] {$N_1$} --  ++(29, 0) node[orbital] {$4s$};
       \draw[glevel] (0, 125mm) node[edge] {}      --  ++(29, 0) node[orbital] {$4p_{1/2}$};
       \draw[glevel] (0, 130mm) node[edge] {$N_3$} --  ++(29, 0) node[orbital] {$4p_{3/2}$};
       \draw[glevel] (0, 135mm) node[edge] {}      --  ++(29, 0) node[orbital] {$4d_{3/2}$};
       \draw[glevel] (0, 140mm) node[edge] {$N_5$} --  ++(29, 0) node[orbital] {$4d_{5/2}$};
       \draw[glevel] (0, 145mm) node[edge] {}      --  ++(29, 0) node[orbital] {$4f_{5/2}$};
       \draw[glevel] (0, 150mm) node[edge] {$N_7$} --  ++(29, 0) node[orbital] {$4f_{7/2}$};

       % m lines


       \draw[xjoin, draw=black, fill=black] (5mm, 150mm) -- ++(5mm, -5mm){};
       \draw[xray, draw=black] (10mm, 145mm)  -- node[seig] {$M_{\alpha}$} (10mm, 120mm)   -- (10mm,  100mm) ;
       \draw[xray, draw=black] (25mm, 145mm)  -- node[seig] {$M_{\beta}$}  (25mm, 105mm)  -- (25mm,  95mm) ;


        \filldraw (5mm, 150mm) circle (.1);    \filldraw (10mm, 145mm) circle (.1); \filldraw (25mm, 145mm) circle (.1);
       % l3 lines

       \draw[xray, draw=black!05, fill=black!05] (35mm,   80mm)  -- node[seig] {$L_l$}          (35mm,  63mm)  -- node[fill=white] {0.003}  (35mm,  50mm) ;
       \draw[xray, draw=black!10, fill=black!10] (50mm,   90mm)  -- node[seig] {$L_{\alpha_2}$}  (50mm,  74mm)  -- node[fill=white] {0.09}  (50mm,  50mm) ;
       \draw[xray, draw=black!90, fill=black!90] (65mm,  100mm)  -- node[seig] {$L_{\alpha_1}$}  (65mm,  85mm)  -- node[fill=white] {0.80}  (65mm,  50mm) ;

       \draw[xjoin, draw=black!20, fill=black!20](75mm, 140mm) -- (80mm, 135mm){};
       \draw[xray, draw=black!20, fill=black!20] (80mm, 135mm)  -- node[seig] {$L_{\beta_2}$}  (80mm, 95mm)  -- node[fill=white] {0.11}  (80mm,  50mm) ;

       \filldraw (35mm, 80mm) circle (.1);    \filldraw (50mm, 90mm) circle (.1);     \filldraw (65mm, 100mm) circle (.1);
       \filldraw (75mm, 140mm) circle (.1);   \filldraw (80mm, 135mm) circle (.1);
       % l2

       \draw[xray, draw=black]                   (105mm, 95mm)  -- node[seig] {$L_{\beta_1}$}  (105mm, 70mm) -- node[fill=white] {0.88}  (105mm, 45mm) ;
       \draw[xray, draw=black!20, fill=black!20] (120mm, 135mm) -- node[seig] {$L_{\gamma_1}$} (120mm, 80mm) -- node[fill=white] {0.09}  (120mm, 45mm) ;

       \filldraw (105mm, 95mm) circle (.1);   \filldraw (120mm, 135mm) circle (.1);
       
       % l1
       \draw[xray, draw=black!70, fill=black!70] (145mm, 85mm)  -- node[seig] {$L_{\beta_4}$}  (145mm, 60mm) -- node[fill=white] {0.32}  (145mm, 40mm) ;
       \draw[xray, draw=black!95, fill=black!95] (160mm, 90mm)  -- node[seig] {$L_{\beta_3}$}  (160mm, 70mm) -- node[fill=white] {0.50}  (160mm, 40mm) ;
       \draw[xray, draw=black!25, fill=black!25] (175mm, 125mm) -- node[seig] {$L_{\gamma_2}$} (175mm, 80mm) -- node[fill=white] {0.08}  (175mm, 40mm) ;
       \draw[xray, draw=black!25, fill=black!25](190mm, 130mm) -- node[seig] {$L_{\gamma_3}$} (190mm, 90mm) -- node[fill=white] {0.10}  (190mm, 40mm) ;
       
       \filldraw (145mm, 85mm) circle (.1);   \filldraw (160mm, 90mm) circle (.1);
       \filldraw (175mm, 125mm) circle (.1);   \filldraw (190mm, 130mm) circle (.1);

       % k
       \draw[xray, draw=black!65, fill=black!65] (215mm, 45mm)  -- node[seig] {$K_{\alpha_2}$}  (215mm,  25mm) -- node[fill=white] {0.29} (215mm,  0mm) ;
       \draw[xray, draw=black]                   (230mm, 50mm)  -- node[seig] {$K_{\alpha_1}$}  (230mm,  35mm) -- node[fill=white] {0.54} (230mm,  0mm) ;
       \draw[xray, draw=black!15, fill=black!15] (245mm, 85mm)  -- node[seig] {$K_{\beta_3}$}   (245mm,  45mm) -- node[fill=white] {0.05} (245mm,  0mm);
       \draw[xray, draw=black!25, fill=black!25] (260mm, 90mm)  -- node[seig] {$K_{\beta_1}$}   (260mm,  55mm) -- node[fill=white] {0.09} (260mm,  0mm);

       \draw[xjoin, draw=black!10, fill=black!10] (270mm, 130mm) -- ++(5mm, -5mm){};
       \draw[xray, draw=black!10, fill=black!10] (275mm, 125mm)  -- node[seig] {$K_{\beta_2}$}  (275mm, 65mm)  -- node[fill=white] {0.03}  (275mm,  0mm) ;


       \filldraw (215mm, 45mm) circle (.1);   \filldraw (230mm, 50mm) circle (.1);
      \filldraw (245mm, 85mm) circle (.1);   \filldraw (260mm, 90mm) circle (.1);
   
      \filldraw (270mm, 130mm) circle (.1);   \filldraw (275mm, 125mm) circle (.1);

       }}
   \end{scope}
%%



\end{tikzpicture}
\end{preview}
\end{document}


